
\chapter{Cei necopți}

\verseref{60}
Noaptea e lungă\\
pentru insul care nu poate dormi.\\
O călătorie e lungă pentru insul obosit.\\
Existența ignorantă e lungă și searbădă\\
pentru cei inconștienți de Adevăr.


\verseref{61}
Neavând niciun companion\\
care să fi călătorit\\
cel puțin așa de departe ca și noi,\\
e mai bine a merge singur\\
decât să-i întovărășești\\
pe cei care rămân nehotărâți.


\verseref{62}
„Acesta e copilul meu,\\
aceasta e averea mea”:\\
astfel de gânduri sunt preocupări\\
ale celor necopți.\\
Dacă suntem incapabili\\
să ne luăm asupra noastră\\
chiar și pe noi înșine,\\
de ce să facem astfel de revendicări?


\verseref{63}
Necoptul care știe că e necopt\\
e câtuși de puțin înțelept;\\
necoptul care socotește că e înțelept\\
e, fără îndoială, necopt.


\verseref{64}
Asemeni lingurilor incapabile să guste\\
din savoarea supei\\
sunt cei nesăbuiți care nu pot vedea adevărul,\\
cu toate că trăiesc\\
toată viața printre cei înțelepți.


\verseref{65}
Asemeni limbii care poate aprecia\\
savoarea supei\\
e unul care poate discerne clar adevărul\\
după doar o scurtă\\
asociere cu cei înțelepți.


\verseref{66}
Făcând rău,\\
fără băgare de seamă,\\
nesăbuiții imprudenți\\
își produc propriile fructe amare.\\
Ei se comportă\\
precum inamicii lor cei mai mari.

\verseref{67}
O faptă nu e bine făcută\\
când la reflecție asupra sa remușcarea răsare:\\
cu lacrimi de supărare\\
insul își culege fructele.


\verseref{68}
O faptă e bine făcută\\
când la reflecție\\
asupra sa nicio remușcare nu răsare:\\
cu bucurie insul își culege fructele.


\verseref{69}
Nesăbuiții percep acțiunile rele\\
ca fiind dulci precum mierea\\
până când văd consecințele.\\
Când își zăresc roadele,\\
nesăbuiții suferă într-adevăr.


\verseref{70}
Chiar și după luni de post aspru,\\
subzistând cu o dietă frugală,\\
un necopt nu poate fi comparat ca valoare\\
cu persoana care vede pur și simplu adevărul.


\verseref{71}
Laptele proaspăt nu se brânzește\\
de îndată;\\
nici faptele rele nu își arată\\
de îndată roadele;\\
oricum, necopții suferă\\
din cauza ignoranței lor\\
la fel cum ar fi arși\\
stând pe cărbuni ascunși în cenușă.


\verseref{72}
Necopții sunt aceia care folosesc prostește\\
orice daruri ar avea,\\
distrugându-și averea acumulată.


\verseref{73-74}
Trufia și râvna necopților crește\\
când revendică pentru ei\\
autoritate, recunoaștere\\
și răsplată necuvenite:\\
falsitatea lor le afectează năzuința,\\
își doresc să fie văzuți\\
drept puternici și deslușitori.



\verseref{75}
Calea care duce la câștig lumesc\\
și calea care duce la Eliberare\\
sunt căi diferite.\\
Văzând acest lucru, discipolul care renunță\\
evită distragerea\\
câștigului și succesului lumesc\\
pentru a sălășlui în solitudine.
