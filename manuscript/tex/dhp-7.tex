
\chapter{Cei treziți}


\verseref{90}
Nu există tensiune\\
pentru cei care au ajuns la capătul călătoriei\\
și au fost eliberați\\
din dezastrul tuturor legăturilor încolăcite.


\verseref{91}
Atenți la nevoile călătoriei,\\
cei pe calea conștiinței,\\
precum lebedele, curg ușor,\\
lăsând în urmă vechile locuri de odihnă.


\verseref{92}
La fel ca păsările care nu lasă urme în aer\\
sunt cei a căror minte nu se agață\\
de tentațiile care le sunt oferite.\\
Centrul atenției lor e starea de eliberare fără reper,\\
care pentru ceilalți e imperceptibilă.


\verseref{93}
Sunt cei care\\
sunt liberi de toate piedicile;\\
nu se îngrijorează pentru mâncare.\\
Centrul atenției lor e starea de eliberare\\
fără reper.\\
La fel ca păsările care, zburând prin aer,\\
fără urme se duc pe cale.


\verseref{94}
Precum caii bine-antrenați de stăpâni\\
sunt cei care și-au domesticit simțurile;\\
având inimile eliberate\\
de mândrie și poluare\\
generează încântare atotcuprinzătoare.


\verseref{95}
Sunt acei care descoperă\\
că pot abandona complet\\
reacțiile confuze\\
și să devină răbdători precum pământul;\\
nemișcați de ură,\\
nezdruncinați precum un stâlp,\\
neperturbați precum un iaz limpede și liniștit.


\verseref{96}
Cei care ajung\\
la starea de libertate perfectă\\
prin înțelegere dreaptă\\
sunt neperturbați\\
în corp, vorbire și minte.\\
Ei rămân nemișcați\\
de vicisitudinile vieții.


\verseref{97}
Cei care cunosc increatul,\\
care sunt liberi și statorniciți,\\
care au abandonat toată râvna,\\
sunt ființele cele mai vrednice.


\verseref{98}
Fie într-o pădure,\\
un oraș sau ținut deschis,\\
încântător e sălașul\\
unuia care acum e deplin liber.


\verseref{99}
Ființele libere de dependența de plăceri senzuale\\
cunosc o formă unică de încântare.\\
Ele caută liniște în refugiile pădurii\\
pe care muritorii de rând le-ar evita.
