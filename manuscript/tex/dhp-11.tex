
\chapter{Old Age}


\verseref{146}
De ce există râs?\\
De ce există bucurie\\
când lumea e în flăcări?\\
Cât timp esți nourat de întuneric\\
nu ar trebui să cauți lumina?


\verseref{147}
Privește îndelung asupra acestui corp\\
odată decorat –\\
obișnuia să atragă atenție,\\
dar acum e doar carne purulentă,\\
un obiect putred.\\
Nu e nici sigur, nici substanțial.


\verseref{148}
Acest corp se istovește cu vârsta;\\
devine o gazdă\\
pentru boală\\
– vulnerabil, fragil,\\
decrepit, o masă în dezintegrare,\\
care în cele din urmă\\
sfârșește în moarte.


\verseref{149}
Ce plăcere mai aduce viața\\
odată ce insul a văzut\\
oase vechi decolorate,\\
aruncate și împrăștiate-n jur?


\verseref{150}
Corpul fizic e alcătuit din oase\\
îmbrăcate-n carne și sânge.\\
Agonisite înăuntrul său\\
sunt descompunere și moarte,\\
mândrie și răutate.


\verseref{151}
Primită ca moștenire de la cei înțelepți\\
e cunoașterea\\
că, deși ceea ce impresionant la exterior\\
își pierde splendoarea,\\
și cu toate că\\
trupurile noastre se vor descompune,\\
adevărul singur va supraviețui degenerărilor.

\verseref{152}
Cât timp nesăbuiții, înaintând în vârstă,\\
pun pe ei precum boii în staulul lor,\\
mintea le rămâne mică.


\verseref{153-154}
Preț de multe vieți am pribegit\\
căutându-l, deși e de negăsit,\\
pe clăditorul de case\\
care mi-a provocat suferința.\\
Dar acum ești zărit\\
și nu vei mai construi în cele ce urmează.\\
Grinzile sunt desprinse și\\
coama acoperișului sfărâmată.\\
Toată râvna a încetat;\\
inima mea e una cu nefăcutul.


\verseref{155}
Cei care, tineri fiind,\\
nici nu aleg o viață în renunțare,\\
nici nu-și câștigă existența cuviincios,\\
sfârșesc precum stârcii bătrâni, descurajați,\\
pe lângă un eleșteu fără pești.


\verseref{156}
Cei care, tineri fiind,\\
nici nu aleg o viață în renunțare,\\
nici nu-și câștigă existența cuviincios,\\
vor sfârși deplângând trecutul,\\
prăbușindu-se precum săgețile consumate\\
care și-au ratat ținta.
