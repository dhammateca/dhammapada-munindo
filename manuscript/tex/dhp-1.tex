
\chapter{Perechile}

% NOTE @mircea: review the PDF for long lines. If the line doesn't fit, break
% and indent as below. Review if it makes sense for the reading.

\verseref{1}
Toate stările care iau ființă\\
sunt determinate\\
de inimă.\\
Inima e cea care îndrumă pe cale.\\
Exact cum roata carului\\
urmează amprenta copitei\\ \vin animalului care-l trage,\\
așa suferința va urma\\
cu siguranță\\
când vorbim sau acționăm impulsiv\\
dintr-o inimă impură.


\verseref{2}
Toate stările care iau ființă\\
sunt determinate\\
de inimă.\\
Inima e cea care îndrumă pe cale.\\
La fel de sigur cum umbra noastră\\
nu ne părăsește niciodată,\\
așa bunăstarea va urma\\
când vorbim sau acționăm\\
cu o inimă pură.


\verseref{3}
Când ne ținem strâns\\
de asemenea gânduri precum\\
„M-au abuzat,\\
m-au maltratat,\\
m-au brutalizat,\\
m-au jefuit”\\
păstrăm ura vie.


\verseref{4}
Dacă ne eliberăm complet\\
de gânduri precum\\
„M-au abuzat,\\
m-au maltratat,\\
m-au brutalizat,\\
m-au jefuit”,\\
ura e cucerită.


\verseref{5}
Niciodată ura nu e prin ură cucerită,\\
ci doar prin disponibilitatea de a iubi.\\
Aceasta e lege eternă.

\verseref{6}
Cei ce sunt litigioși\\
au uitat\\
că murim cu toții.
Pentru cei înțelepți,\\
ce reflectă asupra acestui fapt,\\
nu există cârcoteli.

\verseref{7}
La fel cum rafala unei furtuni\\
poate dezrădăcina\\
un copac firav,\\
așa cel care se ține\\
fără băgare de seamă de plăcere,\\
cel care se răsfață cu mâncare\\
și e nepăsător\\
poate fi dezrădăcinat de Mara.


\verseref{8}
La fel cum rafala unei furtuni\\
nu poate urni un munte de piatră,\\
așa cel care contemplă\\
realitatea corpului,\\
care dezvoltă credință și energie,\\
e nemișcat de Mara.

\verseref{9}
Purtarea robei unuia care a renunțat,\\
în sine, nu-l face pe ins pur.\\
Acei care o poartă,\\
dar le lipsește încă sârguința,\\
sunt lipsiți de băgare de seamă.

\verseref{10}
Fiind posedat de autocontrol,\\
onest și sârguincios în conduită,\\
un ins e astfel demn\\
de roba unuia care a renunțat.


\verseref{11}
Încurcând falsul cu realul,\\
și realul cu falsul,\\
insul suferă de o viață\\
a falsității.


\verseref{12}
Dar văzând\\
falsul ca fals\\
și realul ca real,\\
insul trăiește\\
în perfectul real.


\verseref{13}
Asemeni ploii infiltrându-se\\
printr-un acoperiș prost împăiat,\\
pasiunile nesupuse se preling\\
într-o inimă neîmblânzită.


\verseref{14}
Cum ploaia nu poate penetra\\
un acoperiș bine împăiat,\\
așa pasiunile nu pot intra\\
într-o inimă bine instruită.


\verseref{15}
Când vedem clar\\
propria noastră lipsă de virtute\\
suntem umpluți cu mâhnire;\\
de acum și înainte ne mâhnim.


\verseref{16}
Când apreciem complet beneficiul\\
propriilor noastre fapte pure,\\
suntem plini de bucurie;\\
de acum și înainte\\
are loc o celebrare a bucuriei.

\verseref{17}
De acum și înainte\\
cei care înfăptuiesc rău\\
își creează propria suferință.\\
Preocuparea mentală\\
cu gândul „Am făcut rău”\\
le posedă mintea\\
și cad în haos.


\verseref{18}
De acum și înainte\\
cei care își trăiesc viața bine\\
își duc veacul în fericire.\\
Ei sunt plini\\
de o apreciere naturală a virtuții\\
și trăiesc cu încântare.


\verseref{19}
Cu toate că insul poate cunoaște\\
multe despre Dhamma,\\
dacă insul nu trăiește propice –\\
asemeni unui văcar\\
ce râvnește vitele altuia –\\
insul nu va afla\\
niciunul din beneficiile\\
de a umbla Calea.


\verseref{20}
Cunoscând doar puțin\\
despre Dhamma,\\
dar conformându-se cu toată inima,\\
transformând pasiunile\\
lăcomiei, urii și decepției\\
eliberând toate atașamentele\\
de acum și de apoi,\\
insul cu siguranță va afla singur\\
beneficiile de a umbla Calea.
