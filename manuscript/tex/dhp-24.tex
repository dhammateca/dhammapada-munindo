
\chapter{Râvnă}


\verseref{334}
Râvna descătușată\\
crește precum o târâtoare în pădure.\\
Pierdut în aceasta,\\
insul saltă dintr-o parte într-alta\\
precum o maimuță de copac în căutare de fructe.


\verseref{335}
Cultivarea obiceiurilor\\
precum râvna și agățatul\\
se aseamănă cu fertilizarea\\
buruienilor nocive.


\verseref{336}
Așa cum apa cade de pe o frunză de lotus,\\
la fel supărarea se desprinde de cei\\
eliberați de râvnă toxică.


\verseref{337}
Ploile torențiale pot distruge recolte.\\
Mara te poate distruge pe tine.\\
Astfel te implor –\\
smulge toate rădăcinile râvnei.\\
Te binecuvântez\\
pentru lucrarea aceasta.


\verseref{338}
Dacă rădăcinile nu sunt smulse,\\
un copac crește iar și iar;\\
suferința se întoarce la noi\\
cât timp râvna rămâne.


\verseref{339}
Când șiroaiele plăcerii senzuale se reped\\
pot provoca un torent de năzuință.


\verseref{340}
Șiroaiele râvnei curg peste tot.\\
Târâtoarea sălbatică a râvnei\\
se întinde și prinde în mreje.\\
Prin discernerea târâtoarei cu introspecție,\\
îndepărtați-o.


\verseref{341}
Ființele resimt plăcerea natural;\\
însă când plăcerea este contaminată cu râvnă\\
a nu o elibera duce la frustrare,\\
iar suferința searbădă urmează.


\verseref{342}
Captivi în obiceiurile râvnei,\\
oamenii intră în panică\\
asemenea iepurilor prinși într-o cursă.\\
Reacțiile lor fortifică durerea\\
capturării.


\verseref{343}
Captivi în obiceiurile râvnei,\\
oamenii intră în panică\\
asemenea iepurilor prinși într-o cursă.\\
Dacă doriți să vă eliberați de capturare,\\
râvna însăși trebuie eliminată.


\verseref{344}
Sunt inși care au început să parcurgă\\
calea către libertate,\\
însă pe seama dorinței se întorc la înlănțuire.


\verseref{345-346}
Prin înțelepciune,\\
devine limpede că a fi ținut după gratii\\
sau a fi înlănțuit\\
nu limitează la fel de tare\\
precum dragostea față de posesii\\
și obsesia față de relații.\\
Aceste legături, deși nu la fel de evidente,\\
sunt puternice și ne rețin.\\
A renunța la atașamentul\\
față de lumea simțurilor\\
înseamnă a fi eliberat din temnița râvnei.


\verseref{347}
Asemenea unui păianjen prins\\
în propria pânză,\\
o ființă învăluită în râvnă senzuală\\
trebuie eliberată de propriile năzuințe\\
înainte ca aceasta să poată\\
umbla nestingherită.


\verseref{348}
Renunțați la trecut.\\
Renunțați la viitor.\\
Renunțați la prezent.\\
Cu o inimă liberă,\\
traversați pe țărmul\\
aflat dincolo de suferință.


\verseref{349}
Legăturile obiceiurilor și râvnei amăgitoare\\
sunt întărite când insul își lasă mintea,\\
fără băgare de seamă, să stăruie asupra\\
obiectelor dorinței.


\verseref{350}
Însă cel ce se delectează\\
în potolirea gândurilor senzuale,\\
alert și cultivând conștiința\\
privind aspectele respingătoare ale corpului,\\
se eliberează de râvnă\\
și dezleagă obiceiurile amăgitoare.


\verseref{351}
Nu mai este nevoie de îndreptare\\
pentru cei care și-au atins țelul;\\
sunt eliberați de teamă și năzuință.\\
Spinii existenței au fost scoși.


\verseref{352}
Un călăuzitor este acela care a renunțat\\
la toată râvna și la a se agăța de lume;\\
care a văzut\\
adevărul dincolo de forme,\\
și totuși posedă o cunoaștere desăvârșită\\
a cuvintelor.\\
Se poate spune\\
despre o asemenea ființă măreață\\
că și-a îndeplinit sarcina.


\verseref{353}
Nu declar pe altcineva drept dascălul meu,\\
căci de unul singur am ajuns\\
la înțelepciunea care depășește totul,\\
înțelege totul, renunță la tot.\\
Sunt pe deplin eliberat de toată râvna.


\verseref{354}
Darul adevărului este cel mai mare dintre daruri.\\
Savoarea realității\\
depășește orice alte savoare.\\
Încântarea adevărului\\
transcende toate încântările.\\
Eliberarea de râvnă\\
pune capăt tuturor suferințelor.


\verseref{355}
Bogățiile îi ruinează mai ales pe cei necopți,\\
însă nu și pe cei ce caută eternitatea.\\
La fel cum nu se gândesc la binele celorlalți\\
și provoacă rău,\\
necopții se distrug și pe sine.


\verseref{356}
Buruienile provoacă stricăciuni pe câmpuri.\\
Voluptatea provoacă stricăciuni tuturor ființelor.\\
Sprijiniți-i pe cei eliberați de voluptate;\\
darul va duce la roade însemnate.


\verseref{357}
Buruienile provoacă stricăciuni pe câmpuri.\\
Ura provoacă stricăciuni tuturor ființelor.\\
Sprijiniți-i pe cei eliberați de ură;\\
darul va duce la roade însemnate.


\verseref{358}
Buruienile provoacă stricăciuni pe câmpuri.\\
Confuzia provoacă stricăciuni\\
tuturor ființelor.\\
Sprijiniți-i pe cei eliberați de confuzie;\\
darul va duce la roade însemnate.


\verseref{359}
Buruienile provoacă stricăciuni pe câmpuri.\\
Invidia provoacă stricăciuni tuturor ființelor.\\
Sprijiniți-i pe cei eliberați de invidie,\\
iar roadele vor fi însemnate.
