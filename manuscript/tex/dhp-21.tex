
\chapter{Diverse}


\verseref{290}
Înțelepciunea îngăduie renunțarea\\
la o fericire mai neînsemnată\\
în căutarea unei fericiri mai însemnate.


\verseref{291}
Eșuezi în căutarea fericirii\\
dacă o cauți cu prețul stării de bine\\
a altora.\\
Lațul voinței bolnave\\
te poate strânge în continuare.


\verseref{292}
A lăsa nefăcut\\
ceea ce ar trebui făcut\\
și a face ceea ce ar trebui evitat\\
duce la neatenție și trufie.\\
Iar astfel confuzia va spori.


\verseref{293}
Confuzia încetează\\
susținând cultivarea\\
lucidității cufundate în corp,\\
evitând ceea ce nu ar trebui făcut\\
și făcând lucid\\
ceea ce ar trebui făcut.


\verseref{294}
Prin îndepărtarea râvnei și a îngâmfării,\\
eradicarea concepțiilor greșite\\
și scăparea atașamentelor\\
înșelătoare de sub domnia simțurilor,\\
ființa nobilă merge liber înainte.


\verseref{295}
Odată cu îndepărtarea\\
tuturor obstacolelor către Cale –\\
lăcomie, furie, trândăvie și lene,\\
îngrijorare, anxietate și îndoială –\\
ființa nobilă merge liber înainte.


\verseref{296}
Discipolii lui Buddha sunt pe deplin treji,\\
sălășluind atât ziua, cât și noaptea\\
întru contemplarea Celui Trezit.


\verseref{297}
Discipolii lui Buddha sunt pe deplin treji,\\
sălășluind atât ziua, cât și noaptea\\
întru contemplarea realității.


\verseref{298}
Discipolii lui Buddha sunt pe deplin treji,\\
sălășluind atât ziua, cât și noaptea\\
întru contemplarea comuniunii\\
dintre ființele care s-au trezit.


\verseref{299}
Discipolii lui Buddha sunt pe deplin treji,\\
sălășluind atât ziua, cât și noaptea\\
întru contemplarea naturii\\
adevărate a corpului.


\verseref{300}
Discipolii lui Buddha sunt pe deplin treji\\
atât ziua, cât și noaptea,\\
delectându-se în compasiune.


\verseref{301}
Discipolii lui Buddha sunt pe deplin treji\\
atât ziua, cât și noaptea,\\
delectându-se în cultivarea\\
inimii.


\verseref{302}
O viață a renunțării\\
este greu de trăit;\\
încercările sale\\
sunt greu de socotit drept\\
plăcute.\\
Totuși, și viața de gospodar\\
este greu de trăit;\\
e durere asocierea\\
cu cei alături de care\\
insul nu consideră că merge în tovărășie.\\
Pribegitul fără angajament\\
va fi mereu greu;\\
de ce să nu renunțați\\
la căutarea înșelătoare a\\
durerii?


\verseref{303}
Un călător înzestrat cu virtute,\\
disciplinat și dedicat\\
conduitei corecte,\\
va fi primit cu onoare;\\
un asemenea ins poate fi\\
recunoscut și poate călători\\
încrezător.


\verseref{304}
Cei buni se deslușesc încă\\
de la distanță.\\
Strălucesc asemenea\\
vârfurilor îndepărtate din Himalaya.\\
Cei neinstruiți dispar pur și simplu\\
asemenea săgeților trase în\\
întuneric.


\verseref{305}
Consacrați-vă cu entuziasm\\
cultivării solitudinii –\\
ședeți singuri, dormiți\\
singuri, umblați singuri\\
și delectați-vă ca și cum\\
ați fi izolați în pădure.
