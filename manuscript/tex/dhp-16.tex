
\chapter{Afecțiune}


\verseref{209}
Sunt ființe pe urma\\
a ceea ce ar trebui evitat\\
și care evită\\
ceea ce ar trebui urmărit.\\
Cufundați în simțuri își pierd calea,\\
iar mai apoi îi invidiază\\
pe cei ce cunosc adevărul.


\verseref{210}
A pierde compania celor\\
alături de care ne simțim acasă\\
e dureros,\\
iar asocierea cu cei care ne displac\\
e chiar mai neplăcută;\\
așadar, nu vă lăsați\\
nici în voia companiei\\
celor cu care vă simțiți acasă,\\
nici celor care vă displac.


\verseref{211}
Fiți prevăzători cu atașamentul\\
ce izvorăște din afecțiune,\\
căci despărțirea de cei dragi e dureroasă;\\
atunci când nu vă aliniați\\
nici de partea, nici împotriva afecțiunii,\\
nu vor exista legături care să constrângă.


\verseref{212}
Din îndrăgire izvorăște mâhnirea.\\
Din îndrăgire izvorăște teama de a pierde.\\
Iar dacă insul nu îndrăgește\\
nu există mâhnire –\\
cum ar putea fi prezentă, astfel, teama?


\verseref{213}
A se pierde în afecțiune\\
aduce supărare;\\
a se pierde în afecțiune\\
aduce teamă.\\
Eliberarea de afecțiune\\
înseamnă încetarea supărării –\\
cum ar putea fi prezentă, astfel, teama?


\verseref{214}
A se pierde în desfătare\\
aduce supărare;\\
a se pierde în desfătare\\
aduce teamă.\\
A se elibera de resimțirea oricărei desfătări\\
înseamnă încetarea supărării –\\
cum ar putea fi prezentă, astfel, teama?


\verseref{215}
A se pierde în voluptate\\
aduce supărare;\\
a se pierde în voluptate\\
aduce teamă.\\
A nu se pierde în voluptate\\
înseamnă încetarea supărării –\\
cum ar putea fi prezentă, astfel, teama?



\verseref{216}
A se pierde în râvnă\\
aduce supărare;\\
a se pierde în râvnă\\
aduce teamă.\\
Eliberarea de râvnă  \\
înseamnă încetarea supărării –\\
cum ar putea fi prezentă, astfel, teama?


\verseref{217}
Cei îndrăgiți în mod firesc\\
sunt cei ce trăiesc săvârșind fapte drepte\\
și au descoperit Calea,\\
iar prin introspecție\\
au devenit consacrați adevărului.


\verseref{218}
Cei care tânjesc\\
după ceea ce nu se poate defini,\\
cu inimile pline de inspirație,\\
cu mințile eliberate\\
de râvna senzuală,\\
se numesc\\
„cei destinați eliberării”.


\verseref{219-220}
Așa cum familia și prietenii\\
îi întâmpină cu bucurie\\
pe cei dragi ce se întorc acasă de departe,\\
la fel propriile lor fapte bune\\
îi întâmpină pe cei care le-au făcut\\
atunci când trec de la viața aceasta la următoarea.
