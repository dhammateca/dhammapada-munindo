
\chapter{Impurități}


\verseref{235}
Asemenea unei frunze veștejite,\\
aveți mesagerul morții alături.\\
Chiar dacă urmează o călătorie lungă,\\
încă nu ați făcut nicio provizie.


\verseref{236}
Grăbiți-vă întru cultivarea înțelepciunii;\\
clădiți-vă o insulă.\\
Scăpați de întinare și pângărire,\\
veți deveni ființe nobile.


\verseref{237}
A sosit momentul să vă aflați\\
în prezența domniei sale, Moartea.\\
Nu e răgaz de odihnă în timpul călătoriei,\\
și totuși ce provizii v-ați făcut?


\verseref{238}
Grăbiți-vă întru cultivarea înțelepciunii;\\
clădiți-vă o insulă.\\
Scăpați de întinare și pângărire,\\
veți fi eliberați de naștere și de moarte.


\verseref{239}
Treptat, treptat,\\
clipă cu clipă,\\
înțelepții își înlătură propriile impurități\\
așa cum un aurar înlătură zgura.


\verseref{240}
Așa cum fierul e distrus de rugina pe care o produce,\\
cei care înfăptuiesc rău sunt corodați\\
de propriile acțiuni.


\verseref{241}
Lipsa studiului duce la uitarea învățăturilor;\\
neglijarea atrage stricăciunea căminului;\\
lenea duce la pierderea frumuseții;\\
nebăgarea de seamă nimicește atenția.


\verseref{242}
Desfrânarea\\
îi înjosește pe cei ce o practică;\\
zgârcenia îi înjosește pe cei ce ar putea dărui.\\
Faptele care își înjosesc făptuitorul\\
sunt cu siguranță întinări.


\verseref{243}
Însă cea mai gravă întinare dintre toate este ignoranța.\\
Purificați-vă de aceasta și veți fi liberi.


\verseref{244}
Viața pare să fie ușoară\\
pentru insul căruia îi lipsește rușinea,\\
care este la fel de neobrăzat precum o cioară,\\
arogant, agresiv, băgăreț și viciat.


\verseref{245}
Viața nu este ușoară pentru cei ce au simțul rușinii,\\
cei modești, cu mintea pură și detașați,\\
drepți și contemplativi.


\verseref{246-247}
Insul care distruge viața, nu ține seama de adevăr,\\
iresponsabil sexual,\\
care ia ceea ce de drept nu îi aparține\\
și se lasă fără băgare de seamă în voia drogurilor\\
își distruge tocmai rădăcinile propriei vieți.


\verseref{248}
Insul cufundat în bunătate\\
trebuie să știe cele ce urmează:\\
lipsa stăpânirii de sine e dezastruoasă.\\
Nu permiteți lăcomiei și purtării nepotrivite să vă prelungească agonia.


\verseref{249-250}
People are inspired to be generous\\
according to their faith and trust.\\
If we become discontented\\
with what we have been given,\\
our meditation will be filled\\
with endless mental affliction;\\
but if we are free from this discontent,\\
our meditation is full of  peace.


\verseref{251}
Nu există niciun alt foc precum voluptatea,\\
nicio obstrucție precum ura,\\
nicio capcană precum decepția\\
și nicio vâltoare precum râvna.


\verseref{252}
E ușor a vedea defectele altora,\\
însă e nevoie de curaj pentru a privi la cele proprii.\\
Cum se întâmplă cu pleava,\\
un ins ar putea alege neajunsurile altora,\\
în timp ce și le ascunde pe cele proprii;\\
așa cum un vânător șiret\\
s-ar putea ascunde de pradă.


\verseref{253}
Cei ce sunt mereu în căutarea defectelor altora –\\
prejudecățile lor sporesc și se află departe de libertate.


\verseref{254}
În aer nu se găsesc drumuri,\\
nu există o altă liberare în afară de Cale.\\
Majoritatea se dedau proliferării,\\
de care Cei Treziți sunt eliberați.


\verseref{255}
În aer nu se găsesc drumuri,\\
nu există o altă liberare în afară de Cale.\\
Nu există lucruri condiționate care să fie permanente,\\
și totuși Buddha rămân neperturbați.
