
\chapter{Lumea}


\verseref{167}
Abandonând căile nevrednice\\
și netrăind neglijent,\\
fără a avea perspective greșite,\\
încetăm în a perpetua deziluzia.


\verseref{168}
Nu afișa falsă umilitate.\\
Rămâi ferm raportat la țelul tău.\\
Practicând cu atenția cuvenită,\\
ajungi la împăcarea cu sine\\
atât acum, cât și în viitor.


\verseref{169}
Trăiește-ți viața în bun acord cu Calea –\\
evită o viață confuză.\\
O viață bine trăită duce la împăcarea cu sine\\
atât acum, cât și în viitor.


\verseref{170}
Regele Morții nu-i poate găsi\\
pe cei care privesc lumea\\
ca fiind nesubstanţială,\\
ca fiind tranzientă, o bulă –\\
iluzorie, doar un miraj.


\verseref{171}
Vino, vezi această lume.\\
Vezi-o precum o caleașcă împodobită festiv.\\
Vezi cum nesăbuiții sunt hipnotizați\\
de vedeniile lor;\\
totuși pentru cei înțelepți nu există atașament.


\verseref{172}
Sunt aceia care\\
se trezesc din nebăgarea de seamă.\\
Aduc lumină în lume\\
precum luna\\
ivindu-se dintre nori.


\verseref{173}
Cel care transformă căi vechi\\
și lipsite de băgare de seamă\\
în căi noi și fapte prielnice\\
aduce lumină în lume\\
precum luna eliberată de nori.


\verseref{174}
Dacă păsările sunt prinse într-o plasă\\
doar câteva vor scăpa vreodată.\\
În această lume a iluziei\\
puțini își văd de drum până la eliberare.


\verseref{175}
Lebedele albe se ridică în aer.\\
Yoghinii înzestrați\\
călătoresc prin spațiu.\\
Ființele înțelepte\\
transcend deziluzia lumească\\
amăgind hoardele lui Mara.


\verseref{176}
Pentru insul care încalcă\\
legea veridicității,\\
insul care nu ține seama\\
de existența viitoare,\\
nu există niciun fel de rău\\
care e imposibil.


\verseref{177}
Cei care eșuează în a prețui generozitatea\\
nu ajung pe tărâmurile cerești.\\
Însă cei înțelepți se bucură când oferă\\
și își duc de-a pururi veacul în exaltare.


\verseref{178}
Mai bine decât a conduce întreaga lume,\\
mai bine decât a ajunge în rai,\\
mai bine decât stăpânirea asupra universului,\\
e un angajament ireversibil față de Cale.
