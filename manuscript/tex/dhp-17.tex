
\chapter{Furie}


\verseref{221}
Leapădă-te de furie.\\
Abandonează trufia.\\
Eliberează-te de toate constrângerile.\\
Cei cu inima pură\\
care nu se agață nici de corp, nici de minte\\
nu cad pradă suferinței.


\verseref{222}
Spun că cei care înfrânează furia\\
așa cum un vizitiu controlează\\
un rădvan în viteză\\
dețin complet controlul asupra vieților lor;\\
alții stau doar cu mâinile pe hățuri.


\verseref{223}
Transformă furia cu bunătate\\
și răutatea cu ceea ce e prielnic,\\
răutatea cu generozitate\\
și înșelăciunea cu integritate.


\verseref{224}
Aceste trei căi\\
conduc către paradisuri:\\
a susține adevărul,\\
a rezista în fața furiei\\
și a dărui, chiar dacă ai doar puțin de împărțit.


\verseref{225}
Cei Treziți nu provoacă rău.\\
Se bucură de abținere dreaptă\\
și se îndreaptă către imuabilitate\\
acolo unde nu se mai mâhnesc.


\verseref{226}
Toate impuritățile dispar\\
din mințile celor mereu vigilenți,\\
care se instruiesc zi și noapte,\\
și ale căror vieți sunt dedicate pe deplin liberării.


\verseref{227}
Încă din timpurile străvechi\\
cei care vorbesc prea mult sunt criticați,\\
precum și cei care vorbesc prea puțin\\
și cei care nu vorbesc deloc.\\
Toți cei din această lume sunt criticați.


\verseref{228}
Nu a trăit vreodată,\\
nici nu va trăi,\\
nici nu trăiește în prezent \\
un om care să fie doar blamat\\
sau elogiat pe de-a-ntregul.


\verseref{229}
Cei care trăiesc impecabil,\\
cei deslușitori,\\
inteligenți și virtuoși –\\
aceștia sunt elogiați neîntrerupt de cei înțelepți.


\verseref{230}
Cine i-ar putea învinovăți pe aceia\\
a căror ființă se aseamănă cu aurul?\\
Până și zeii îi elogiază.


\verseref{231}
Fiți prevăzători cu mișcările stângace\\
și purtați-vă trupul cu conștiință.\\
Renunțați la toate acțiunile nefaste\\
și cultivați-le pe cele integre.


\verseref{232}
Fiți prevăzători cu enunțurile născocite\\
și vorbiți mereu cu conștiință.\\
Renunțați la toate cuvântările viclene\\
și cultivați-le pe cele integre.


\verseref{233}
Fiți prevăzători cu gândirea nefastă\\
și stăruiți mereu cu conștiință.\\
Renunțați la toate gândurile insolente\\
și cultivați-le pe cele benefice.


\verseref{234}
Cu pricepere se abțin înțelepții,\\
în acțiune, în gând\\
și în cuvântări.
