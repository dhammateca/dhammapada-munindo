
\chapter{Buddha}


\verseref{179}
Perfecțiunea lui Buddha e completă;\\
nu mai există muncă de făcut.\\
Nu există măsură pentru înțelepciunea lui;\\
limitele sunt de negăsit.\\
În ce fel poate fi el distras\\
de la adevăr?


\verseref{180}
Perfecțiunea lui Buddha e completă;\\
în el nu există râvnă\\
care să-l poată trage în jos.\\
Nu există măsură pentru înțelepciunea lui;\\
limitele sunt de negăsit.\\
În ce fel poate fi el distras\\
de la adevăr?


\verseref{181}
Ființele celeste îi prețuiesc\\
pe Cei Treziți\\
care au văzut Calea pe deplin,\\
care sunt devotați meditației\\
și se încântă în pacea\\
renunțării.


\verseref{182}
Nu e ușor să fii născut ca ființă omenească\\
și să trăiești această viață de muritor.\\
Nu e ușor a găsi oportunitatea\\
de a auzi Dhamma\\
și e rar ca un Buddha să apară.


\verseref{183}
Înfrână-te de la a face rău,\\
cultivă ceea ce e bun,\\
purifică-ți inima.\\
Aceasta e Calea Celor Treziți.


\verseref{184}
Cel care a renunțat\\
nu oprimă pe nimeni.\\
Răbdarea perseverentă\\
e ascetismul superior.\\
Eliberarea profundă,\\
spun Buddha,\\
e țelul suprem.


\verseref{185}
A nu insulta, a nu vătăma,\\
cultivarea abținerii,\\
cu respect față de disciplină,\\
mâncatul cu modestie și împăcarea\\
cu locul unde insul sălășluiește,\\
devotament față de intenția lucidă:\\
aceasta e învățătura lui Buddha.


\verseref{186-187}
Nu în avere mare\\
e împăcarea cu sine,\\
nici în plăceri senzuale,\\
grosolane sau rafinate.\\
Dar în dispariția râvnei\\
e de găsit bucuria\\
de către un discipol al lui Buddha.


\verseref{188-189}
Spre multe locuri ființele se retrag\\
pentru a scăpa de frică:\\
în munți, păduri,\\
câmpii și grădini;\\
la fel și locuri sfinte.\\
Dar niciunul dintre aceste locuri\\
nu oferă refugiu adevărat,\\
niciunul dintre ele nu ne poate elibera de frică.


\verseref{190-191}
Insul care-și găsește refugiul în Buddha\\
în Dhamma și în Sangha\\
vede cu introspecție penetrantă:\\
suferința, cauza ei, izbăvirea ei\\
și Calea care duce spre libertate adevărată.


\verseref{192}
Buddha, Dhamma și Sangha:\\
acestea sunt refugiile adevărate;\\
acestea sunt supreme;\\
acestea duc la Eliberare.


\verseref{193}
E greu a găsi\\
o ființă cu înțelepciune măreață;\\
rare sunt locurile\\
în care se naște.\\
Cei aflați în preajma sa\\
când apare\\
au într-adevăr noroc.


\verseref{194}
Binecuvântată e apariția unui Buddha;\\
binecuvântată e revelarea Dhammei;\\
binecuvântată e concordia Sanghăi;\\
încântătoare e comuniunea armonioasă.


\verseref{195-196}
Nemăsurat e beneficiul\\
obținut din cinstirea celor\\
ce sunt puri și dincolo de frică.\\
Ființele care au descoperit eliberarea\\
de supărare și mâhnire\\
sunt demne de onoare.
