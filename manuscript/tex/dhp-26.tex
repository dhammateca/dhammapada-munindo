
\chapter{Ființă măreață}


\verseref{383}
Îndepărtați cu sârguință\\
șiroiul râvnei\\
și abandonați năzuințele senzuale;\\
cunoscând limitările intrinseci\\
ale tuturor lucrurilor formate,\\
înțelegeți ceea ce este de necreat.


\verseref{384}
Toate lanțurile limitării cad la pământ\\
pentru cei ce văd cu limpezime\\
dincolo de cele două dhamma.


\verseref{385}
Spun că o ființă este măreață\\
atunci când nu stă nici pe țărmul acesta,\\
pe țărmul celălalt,\\
nici pe vreun alt țărm.\\
O asemenea ființă este liberă de toate legăturile.


\verseref{386}
Spun că o ființă este măreață\\
atunci când sălășluiește izolată și în tihnă,\\
știindu-și inima eliberată de toate impuritățile,\\
cu sarcina încheiată,\\
purificată de toate tendințele compulsive\\
și trează.


\verseref{387}
Soarele strălucește ziua,\\
luna strălucește noaptea.\\
Însă atât toată ziua, cât și toată noaptea\\
Buddha strălucește\\
cu o splendoare glorioasă.


\verseref{388}
Odată cu transformarea răului,\\
insul se numește ființă măreață.\\
Atunci când trăiește pașnic,\\
insul se numește contemplator.\\
Odată cu renunțarea la prihănire,\\
insul se numește cel ce renunță.


\verseref{389}
Neluarea revanșei\\
este specifică ființelor mărețe.\\
Nu se lasă cuprinse de furie.\\
Dacă sunt atacate,\\
nu le stă în fire să contraatace.


\verseref{390}
Suferința încetează\\
în măsura în care vă eliberați de intenția de a face rău.\\
Nu există măreție autentică\\
acolo unde nu există reținerea furiei.


\verseref{391}
Insul care se abține de la a face rău\\
cu trupul, vorba sau mintea\\
se poate numi ființă vrednică.


\verseref{392}
Devotamentul și respectul\\
trebuie oferite celor care ne-au arătat Calea.


\verseref{393}
Insul nu trebuie considerat a fi\\
demn de respect\\
pentru că s-a născut sau pentru formarea sa\\
sau orice alt semn exterior;\\
puritatea și înțelegerea adevărului\\
stabilesc valoarea unui ins.


\verseref{394}
Podoabele voastre exterioare\\
și preocupările pseudospirituale\\
sunt irelevante dacă pe dinăuntru\\
rămâneți neglijenți.


\verseref{395}
A nu se preocupa de aspectul exterior,\\
însă a se dedica cultivării intense și neobosite –\\
aceasta înseamnă măreția.

\verseref{396}
Niciun ins nu este nobil\\
numai datorită a ceea ce moștenește.\\
Nobilitatea provine din curățarea sinelui\\
de toate impuritățile și atașamentele.

\verseref{397}
Pe cel care a tăiat toate funiile\\
și a descoperit curajul,\\
cel care este mai presus de atașamente și pângărire,\\
pe acela îl recunosc drept ființă măreață.


\verseref{398}
Pe cel ce desface legăturile urii,\\
dezleagă frânghiile râvnei,\\
descuie lacătele perspectivei greșite,\\
deschide ușile ignoranței și vede adevărul,\\
pe acela îl recunosc drept ființă măreață.


\verseref{399}
Puterea răbdării\\
este tăria ființelor nobile;\\
acestea pot fi încătușate,\\
pot îndura abuz verbal și bătăi\\
fără ranchiună.

\verseref{400}
Pe cei eliberați de furie,\\
disciplinați cu simplitate, virtuoși,\\
instruiți corect\\
și mai presus de renaștere,\\
pe aceia îi numesc ființe mărețe.


\verseref{401}
Așa cum apa alunecă de pe o frunză de lotus,\\
plăcerile senzuale\\
nu se prind de o ființă măreață.


\verseref{402}
Pe cei care cunosc libertatea\\
de a fi lăsat deoparte\\
povara atașamentului față de corp-minte,\\
pe aceia îi numesc ființe mărețe.


\verseref{403}
Pe cei ce posedă o înțelepciune profundă,\\
care văd ce anume este în conformitate cu Calea și ce nu este,\\
pe cei care au făcut tot ce le stă în putință,\\
pe aceia îi numesc ființe mărețe.


\verseref{404}
Cei eliberați de atașamentul\\
fie față de laici, fie față de alții care au renunțat,\\
și astfel hoinăresc fără dorințe\\
sau fără grijă față de siguranța de orice fel\\
sunt ființe mărețe.


\verseref{405}
Cei care au renunțat\\
la a folosi forța în relațiile cu alte ființe,\\
fie slabe, fie puternice,\\
care nici nu omoară,\\
nici nu provoacă omorârea,\\
aceia se pot numi ființe mărețe.


\verseref{406}
Cei ce rămân prietenoși printre cei ostili,\\
pașnici printre cei agresivi\\
și care nu se atașează de lucrurile\\
de care alții depind\\
sunt ființe mărețe.


\verseref{407}
Pentru o ființă măreață,\\
voluptatea și rea-voința,\\
aroganța și îngâmfarea cad\\
așa cum chiar și cea mai mică sămânță\\
ar cădea de pe vârful unui ac.


\verseref{408}
Pe cei ce spun adevărul\\
și încurajează cu blândețe,\\
fără a se întrece cu cineva,\\
pe aceia îi numesc ființe mărețe.


\verseref{409}
Acțiunile unei ființe mărețe sunt pure.\\
Ființele mărețe nu își însușesc niciodată conștient\\
ceea ce aparține altuia.

\verseref{410}
Inima unei ființe mărețe este liberă.\\
Ființele mărețe nu mai tânjesc\\
după lucrurile din această lume\\
sau din orice altă lume.


\verseref{411}
Inima unei ființe mărețe este liberă.\\
Cu o înțelegere precisă,\\
dincolo de îndoială,\\
ființele mărețe și-au înfipt bine picioarele\\
în tărâmul liberării.


\verseref{412}
Oricine a depășit\\
toate legăturile cu binele și răul,\\
este purificat și eliberat de supărare\\
trebuie numit o ființă măreață.


\verseref{413}
Ființele eliberate de năzuință\\
și care, asemenea lunii pe un cer fără de nori,\\
sunt pure, calme și senine:\\
pe acele ființe le numesc mărețe.


\verseref{414}
Există ființe\\
care parcurg drumul greu\\
de-a lungul mlaștinii periculoase\\
a pasiunilor pângăritoare,\\
traversează oceanul decepției\\
prin întunericul ignoranței\\
și trec dincolo.\\
Sunt întărite de contemplare înțeleaptă,\\
sigure fără de îndoieli, eliberate;\\
aceste ființe sunt cu adevărat mărețe.


\verseref{415}
O ființă măreață\\
este aceea care, abținându-se de la dorințele\\
plăcerii senzuale,\\
trăiește o viață fără de adăpost\\
și se eliberează\\
atât de dorința senzuală,\\
cât și de devenirea perpetuă.


\verseref{416}
O ființă măreață\\
este aceea care, abținându-se de la orice urmă de râvnă,\\
trăiește în pribegie\\
și se eliberează\\
atât de râvnă,\\
cât și de devenirea perpetuă.


\verseref{417}
O ființă măreață\\
este aceea care poate vedea dincolo de toate\\
delectările evidente,\\
precum și de plăcerile subtile\\
pentru a se elibera de atașament.


\verseref{418}
Pe insul care a încetat să urmărească\\
plăcerile și neplăcerile,\\
liniștit,\\
de neclintit de condițiile lumești –\\
pe acela îl numesc ființă măreață.


\verseref{419}
Spun că oricine înțelege pe deplin\\
trecerea și reapariția ființelor,\\
oricine rămâne conștient,\\
neatașat, treaz,\\
conștient de acțiunea corectă,\\
acela este o ființă măreață.


\verseref{420}
Starea de după moarte a ființelor mărețe\\
este de nedeslușit;\\
nu rămâne nicio urmă de pasiune.\\
Acestea sunt pure.


\verseref{421}
Oricine trăiește eliberat\\
de obiceiurile agățării\\
de trecut, prezent sau viitor\\
fără atașamente față de nimic\\
este o ființă măreață.


\verseref{422}
O ființă măreață\\
este la fel de netemătoare precum un taur,\\
nobilă, puternică, înțeleaptă, sârguincioasă;\\
vede dincolo de decepții\\
și este curată, atentă și trează.


\verseref{423}
A înțelege toate dimensiunile\\
existențelor trecute,\\
a vedea cu acuratețe toate tărâmurile,\\
a ajunge la finalul renașterilor,\\
a ști cu introspecție\\
ceea ce trebuie știut,\\
a elibera inima de ignoranță –\\
aceasta înseamnă cu adevărat a fi fost făcut măreț.
