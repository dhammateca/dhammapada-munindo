
\chapter{Mintea}


\verseref{33}
La fel cum un făurar\\
dă forma unei săgeți,\\
așa cei înțelepți își cultivă mintea,\\
atât de excitabilă, incertă\\
și dificil de controlat.

\verseref{34}
Asemeni unui pește care\\
fiind târât din casa-i în apă\\
și aruncat pe teren uscat\\
se va zbate,\\
astfel inima va tremura\\
când se retrage din curentul Marei.


\verseref{35}
Mintea activă e dificil de domesticit,\\
zburdalnică și hoinărind oriunde vrea;\\
domesticirea ei e esențială,\\
călăuzind la bucuria bunăstării.


\verseref{36}
Mintea protejată și păzită\\
duce la trai ușor.\\
Cu toate că mintea aceasta e subtilă,\\
evazivă și dificil de văzut,\\
insul care e atent\\
trebuie să se îngrijească\\
și să vegheze asupra ei.


\verseref{37}
Hoinărind de capul ei de colo colo,\\
fără formă,\\
mintea se așterne în caverna inimii din interior.\\
Să pui stăpânire pe ea\\
e eliberare din înlănțuirea ignoranței.


\verseref{38}
În acela a cărui minte e nestatornică,\\
a cărui inimă e nepregătită\\
cu învățături adevărate,\\
a cărui credință nu e matură,\\
plenitudinea înțelepciunii încă nu se manifestă.


\verseref{39}
Nu există frică\\
dacă inima e necontaminată\\
de pasiuni\\
și mintea e liberă de voință bolnavă.\\
Văzând dincolo de bine și rău,\\
acela e treaz.


\verseref{40}
Văzând acest corp fragil\\
ca un vas de lut,\\
și fortificând inima precum zidurile unui oraș,\\
insul o poate confrunta pe Mara\\
cu arma introspecției.\\
Având avantajul detașării,\\
insul protejează ce a fost deja dobândit.


\verseref{41}
Cu siguranță acest corp\\
curând va sta fără viață\\
aruncat\\
încolo pe sol\\
vidat de conștiință\\
și la fel de inutil\\
precum un butuc ars.


\verseref{42}
Mai mult decât un hoț,\\
Mai mult decât un inamic,\\
o inimă îndrumată greșit\\
îl duce pe ins la necaz.


\verseref{43}
Nici mama, nici tata\\
nici un alt membru al unei familii\\
nu îți poate oferi binecuvântările\\
generate de propria inimă bine îndrumată.
