
\chapter{Sinele}


\verseref{157}
Dacă ținem la noi,\\
atunci ne dăm în propria paza\\
atât ziua, cât și noaptea.


\verseref{158}
E înțelept\\
a ne îndrepta pe sine\\
înainte de a-i instrui pe alții.


\verseref{159}
Sinele insului e cel mai greu de disciplinat.\\
Să faci ceea ce spui:\\
îmblânzește-te\\
înainte de a încerca să-i îmblânzești pe alții.


\verseref{160}
Sincer suntem noi\\
cei de care depindem;\\
cum am putea realmente\\
să depindem de altcineva?\\
Când ajungem\\
la starea de autosuficiență\\
găsim un refugiu rar.


\verseref{161}
La fel cum diamantul poate tăia\\
prin piatra ce odată l-a adăpostit,\\
astfel și propriul rău te poate măcina.


\verseref{162}
Răufăcătorii dedicați\\
se comportă față de ei\\
precum inamicii lor cei mai mari.\\
Sunt precum târâtoarele\\
care strangulează copacii\\
care le suștin.


\verseref{163}
E ușor a face ce e fără de real folos\\
pentru sine,\\
dar e firește dificil să faci\\
ceea ce e într-adevăr benefic și bun.


\verseref{164}
Precum bambusul\\
care se distruge dând roade,\\
la fel se rănesc și cei necopți\\
având perspective greșite\\
și luându-i în derâdere pe cei vrednici\\
care trăiesc în armonie cu Calea.


\verseref{165}
Noi înșine facem rău\\
și noi înșine suntem făcuți impuri.\\
Noi înșine evităm răul\\
și noi înșine suntem făcuți puri.\\
Aspectul purității e grija noastră.\\
Nimeni altcineva nu poate fi responsabil.


\verseref{166}
Cunoscând Calea pentru sine,\\
parcurge-o temeinic.\\
Nu îngădui nevoilor altora,\\
oricât de insistente,\\
să ivească confuzie.
