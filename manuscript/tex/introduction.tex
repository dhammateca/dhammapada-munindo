
\chapter{Introducere}

Cartea pe care o ai în față e un coș strălucitor de lumină, iluminând din plin condiția umană. E o versiune a operei clasice budiste Dhammapada; nu e o traducere mot à mot, ci o interpretare liberă cu scopul de a comunica spiritul viu al textului, nestânjenită de atașamentul față de exactitatea formală. Intenția autorului, Ajahn Munindo, a fost să prezinte o versiune contemporană a textului pe care cititorii să o folosească în investigația lor a Căii. Prin urmare titlul este „O Dhammapada pentru contemplație”, sugerând că lucrarea nu trebuie considerată o traducere categorică, ci o invitație la a întâlni și contempla înțelepciunea lui Buddha.

Contextul acestei interpretări e o viață a contemplării. Ajahn Munindo, născut în Noua Zeelandă, a fost hirotonisit ca bhikkhu sau călugăr al tradiției Theravada din Thailanda în urmă cu douăzeci și cinci de ani. El s-a stabilit acum într-o mică comunitate spirituală în Northumberland, Anglia, unde trăiește sub auspiciul unui cod de disciplină care își are originile din vremurile lui Buddha, ce încurajează simplitatea și atenția corectă. Transplantarea acestui mod de viață budist monahal din Orient – unde budismul e adânc înrădăcinat – în Vest a însemnat un proces de traducere care implică limbaj, obiceiuri și ritualuri, astfel încât tradiția Theravada, alături de altele, e stabilită acum în ambientul cultural al Occidentului. Prin urmare, „O Dhammapada pentru contemplație”, cu toate că e o interpretare liberă, după standarde scolastice, urmărește să comunice o traducere precisă a valorilor – valorile spirituale ale tradiției budiste vii.

\bigskip

{\raggedleft
Thomas Jones Ph.D.\\
Cambridge, aprilie 2000
\par}
