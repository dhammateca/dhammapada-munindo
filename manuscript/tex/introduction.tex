
\chapter{Introduction}

The book you hold is a sparkling basket of light, full of illumination of the human situation. It is a version of the Buddhist classic, the Dhammapada; not a line-by-line translation but a free rendering that aims to communicate the living spirit of the text, unencumbered by rigid adherence to formal exactness. The intention of the author, Ajahn Munindo, was to present a contemporary version of the text for readers to use in their investigation of the Way. Hence its title is \emph{A Dhammapada for Contemplation}, indicating that the work is not to be considered as a definitive translation, but as an invitation to encounter and to contemplate the Buddha’s wisdom.

It is a life of contemplation that is the context for this rendering. Ajahn Munindo, who was born in New Zealand, was ordained as a bhikkhu or monk of the Theravadan tradition in Thailand twenty-five years ago. He now resides in a small spiritual community in Northumberland, England, where he lives under a code of discipline going back to the Buddha, which encourages simplicity and right attention. Transplanting this Buddhist monastic way of life from the eastern countries where Buddhism has been long established into the West has meant a process of translation involving language, practices and rituals, such that the Theravadan tradition, along with others, is now established within the Western cultural ambience. \emph{A Dhammapada for Contemplation}, therefore, although a free rendering by scholarly standards, aims to communicate a precise translation of values — the spiritual values of the living Buddhist tradition.

\bigskip

{\raggedleft
Thomas Jones Ph.D.\\
Cambridge, April 2000
\par}
