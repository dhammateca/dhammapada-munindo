
\chapter{Cei înțelepți}


\verseref{76}
Doar binecuvântări pot răsări din căutarea\\
companiei persoanelor înțelepte și deslușitoare,\\
care oferă iscusit\\
atât mustrare, cât și sfat\\
ca și cum ar ghida insul\\
spre o comoară ascunsă.


\verseref{77}
Fie ca înțelepții să îndrume ființele\\
departe de întuneric,\\
să dea direcție și sfaturi.\\
Vor fi prețuiți de către cei virtuoși\\
și îndepărtați de către cei necopți.


\verseref{78}
Nu căuta compania prietenilor prost îndrumați;\\
fii prevăzător cu companionii degenerați.\\
Caută compania prietenilor bine îndrumați,\\
cei care sprijină introspecția.


\verseref{79}
Capitularea sinelui Dhammei\\
conduce la existență senină.\\
Înțelepții se delectează perpetuu în adevărul\\
predat de către Unul Trezit.


\verseref{80}
Cei care construiesc canale\\
canalizează cursul apei.\\
Săgetarii fac săgeți.\\
Lemnarii făuresc lemn.\\
Înțelepții se domesticesc.


\verseref{81}
Cum piatra solidă\\
e nemișcată de vânt,\\
așa sunt și cei cu înțelepciunea neperturbată\\
nici de laudă, nici de ocară.


\verseref{82}
La auzul învățăturilor adevărate\\
inimile celor receptivi\\
devin senine,\\
ca un lac adânc, limpede și nemișcat.


\verseref{83}
Ființele virtuoase sunt neatașate.\\
Nu se înfruptă din vorbit fără noimă\\
despre plăceri senzuale.\\
Cunosc atât fericirea,\\
cât și supărarea,\\
dar nu sunt posedate de niciuna.


\verseref{84}
Nici pentru binele propriu,\\
nici în numele altuia\\
o persoană înțeleaptă nu face pagubă\\
– nu pentru hatârul familiei, averii sau câștigului.\\
Unul astfel e de drept numit\\
just, virtuos și înțelept.


\verseref{85}
Puțini sunt cei care ajung departele.\\
Mulți pășesc fără sfârșit înapoi și înainte,\\
neîndrăznind să riște călătoria.


\verseref{86}
Deși e dificil să treci dincolo\\
de marea bătută de ploile pasiunii,\\
cei care trăiesc în conformitate\\
cu Calea bine-predată\\
ajung departele.


\verseref{87-88}
Cu imaginea liberării ca țel\\
cei înțelepți abandonează întunericul\\
și prețuiesc lumina,\\
lasă siguranța măruntă în urmă\\
și caută libertatea de atașament.\\
A fi pe urmele unei asemenea desfaceri\\
e dificil și rar,\\
totuși cei înțelepți o vor căuta,\\
detașându-se de obstrucții,\\
purificând inima și mintea.


\verseref{89}
Eliberându-se de năzuință,\\
neîmpiedicați de obișnuințe deprinse,\\
cei care se aliniază cu Calea\\
se delectează în detașare\\
și, cât încă stau în lume,\\
sunt radioși.
