
\chapter{Calea}


\verseref{273}
Calea cu opt brațe\\
este cel mai onorabil drum,\\
cele patru Adevăruri Nobile\\
cele mai onorabile enunțuri,\\
libertatea de râvnă\\
cea mai onorabilă stare,\\
iar Buddha atotvăzător\\
cea mai onorabilă ființă.


\verseref{274}
Aceasta este singura Cale;\\
nu există alta\\
care să conducă la o viziune limpede.\\
Urmați această Cale,\\
iar Mara va fi dezorientată.


\verseref{275}
Dacă umblați pe cale\\
veți ajunge la capătul suferinței.\\
Văzând chiar eu acest lucru,\\
proclam Calea care înlătură\\
toți spinii.


\verseref{276}
Cei Treziți\\
nu pot decât să indice calea;\\
efortul trebuie să îl depunem noi înșine.\\
Cei ce reflectă cu înțelepciune\\
și intră pe cale sunt eliberați\\
din lanțurile Marei.


\verseref{277}
„Toate lucrurile condiționate sunt\\
nepermanente”;\\
când vom vedea acest lucru prin introspecție\\
ne vom sătura de această viață a suferinței.\\
Aceasta este Calea către purificare.


\verseref{278}
„Toate lucrurile condiționate\\
sunt incomplete”;\\
când vom vedea acest lucru prin introspecție\\
ne vom sătura de această viață a suferinței.\\
Aceasta este Calea către purificare.


\verseref{279}
„Toate realitățile\\
sunt lipsite de un sine statornic”;\\
când vom vedea acest lucru prin introspecție\\
ne vom sătura de această viață a suferinței.\\
Aceasta este Calea către purificare.


\verseref{280}
Dacă, pe când încă sunteți țineri și în putere,\\
amânați atunci când trebuie să acționați,\\
dedându-vă fanteziilor fără băgare de seamă,\\
Calea și înțelepciunea sa\\
nu vor fi niciodată limpezi.


\verseref{281}
Vorbiți cu atenție,\\
gândiți cu abținere,\\
și acționați fără cusur.\\
Purificarea acestor trei mijloace de a vă purta\\
vă va conduce pe Calea înțelepților.


\verseref{282}
A contempla viața conduce la înțelepciune;\\
înțelepciunea apune fără contemplare.\\
Recunoașteți cum se cultivă\\
și distruge înțelepciunea\\
și umblați pe Calea înălțării.


\verseref{283}
Îndepărtați pădurile râvnei,\\
însă fără a ataca și distruge copacii.\\
Îndepărtați întreaga pădure a râvnei\\
și veți vedea Calea către libertate.


\verseref{284}
Cât timp atracția sexuală\\
nu a fost îndepărtată\\
și dacă nu dispare chiar și\\
cea mai mică urmă,\\
inima rămâne dependentă\\
asemenea unui vițel alăptat\\
de vacă.


\verseref{285}
Îndepărtați înlănțuirile afecțiunii\\
așa cum ați smulge o floare de toamnă.\\
Umblați pe Calea care conduce către liberare,\\
elucidată de Cel Trezit.


\verseref{286}
Un necopt\\
este cel ce se dedă visării\\
privind cel mai comod loc\\
de trai, spunând\\
„Aici va fi cald,\\
acolo va fi răcoare” –\\
inconștient de iminența morții.


\verseref{287}
Așa cum o inundație\\
poate mătura din cale un sat întreg,\\
cei cufundați în relații și posesii\\
vor fi purtați de moarte.


\verseref{288-289}
Pe când vă apropiați de moarte\\
nu vă va proteja niciunul\\
dintre atașamentele îndrăgite.\\
Vedeți asta, iar apoi,\\
Cu abținere înțeleaptă și efort de neclintit,\\
grăbiți-vă să vă croiți drumul către liberare.
