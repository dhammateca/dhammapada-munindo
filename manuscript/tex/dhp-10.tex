
\chapter{Agresiune}


\verseref{129}
Având empatie pentru alții,\\
insul vede că toate ființele se tem\\
de pedeapsă și de moarte.\\
Știind asta,\\
insul nu atacă sau provoacă un atac.


\verseref{130}
Având empatie pentru alții,\\
insul vede că toate ființele\\
iubesc viața și se tem de moarte.\\
Știind asta,\\
insul nu atacă sau nu provoacă un atac.


\verseref{131}
A răni ființe vii\\
care, la fel ca noi, caută satisfacția,\\
e să ne aducem rău nouă înșine.


\verseref{132}
A evita rănirea ființelor vii\\
care, la fel ca noi, caută satisfacția,\\
e să ne aducem fericirea.


\verseref{133}
Evită să vorbești aspru altora;\\
vorbitul aspru cere revanșă.\\
Cei răniți de cuvintele tale ar putea să te rănească la rândul lor.


\verseref{134}
Dacă ți se vorbește aspru,\\
fă-te tăcut ca un gong crăpat;\\
a nu căuta răzbunare e un semn al libertății.


\verseref{135}
Precum un păstor\\
conduce vitele spre pășune,\\
bătrânețea și moartea\\
îndreaptă ființele vii.


\verseref{136}
Deși cât timp își îndeplinesc faptele rele\\
nu realizează ce fac,\\
nesăbuiții îndură rezultatele\\
acțiunilor lor în consecință,\\
la fel cum insul se arde când mânuiește foc.


\verseref{137-140}
A pricinui rău\\
celor lipsiți de apărare\\
aduce curând suferință celor care atacă.\\
Vor culege durere sau sărăcie sau pierdere,\\
boală, nebunie sau persecuție,\\
abuz, dezastru sau devastare,\\
și, singuri, după moarte\\
vor trebui să-și confrunte fărădelegile.


\verseref{141}
Nici ascetismul extern,\\
nici înjosirea de sine,\\
nici privarea fizică de orice fel\\
nu-i pot purifica unuia inima\\
încă întunecată de îndoială.


\verseref{142}
Înfățișarea extravagantă exterioară\\
nu constituie în sine un blocaj\\
spre libertate.\\
A avea o inimă în pace,\\
pură, asumată,\\
trează și nevinovată\\
îl distinge pe cel care a renunțat,\\
un drumeț, o ființă nobilă.


\verseref{143}
Un cal bine dresat\\
nu dă motiv de strunire.\\
Rare sunt acele ființe care,\\
prin modestie și disciplină,\\
nu dau motiv de mustrare.


\verseref{144}
Fie ca teroarea\\
de mediocritate nesfârșită,\\
să te zorească spre efort măreț,\\
asemeni unui cal bine-antrenat\\
încurajat de simpla atingere a biciului.\\
Leapădă-te de povara\\
zbuciumului fără sfârșit\\
cu încredere fără regrete,\\
cu puritate în fapte, efort, concentrare\\
și prin angajament conștiincios și disciplinat căii.


\verseref{145}
Cei care construiesc canale\\
canalizează cursul apei.\\
Săgetarii fac săgeți.\\
Lemnarii făuresc lemn.\\
Cei buni se domesticesc.
