\chapter{Rău}


\verseref{116}
Mergi iute spre a face ceea ce e benefic.\\
Abține-ți mintea de la fapte rele.\\
Mintea care e înceată în a face bine\\
poate ușor să găsească plăcere în facerea de rău.


\verseref{117}
Dacă săvârșești o faptă rea,\\
atunci nu o repeta.\\
Evită a găsi plăcere în amintirea ei.\\
Urmarea facerii de rău e dureroasă.


\verseref{118}
Odată săvârșită o faptă benefică\\
e bine să o repeți iar și iar.\\
Fii interesat de plăcerea binefacerii.\\
Fructul bunătății acumulate e împăcarea cu sine.


\verseref{119}
Chiar și cei care înfăptuiesc rău\\
pot trăi în bunăstare\\
atâta vreme cât acțiunile lor\\
încă nu și-au produs direct fructele.\\
În orice caz, când rezultatele\\
faptelor se maturează,\\
consecințele dureroase\\
nu pot fi evitate.


\verseref{120}
Chiar și cei care trăiesc vieți tihnite\\
pot cunoaște suferința\\
atâta vreme cât acțiunile lor\\
încă nu și-au produs direct fructele.\\
În orice caz, când fructele\\
faptelor se maturează,\\
consecințele vesele\\
nu pot fi evitate.


\verseref{121}
Nu ignora efectele răului,\\
zicând: „Nu se va alege nimic din asta.”\\
La fel cum prin căderea treptată\\
a picurilor de ploaie\\
ulciorul de apă se umple,\\
astfel, în timp, cei nesăbuiți sunt corupți de facerea de rău.

\verseref{122}
Nu ignora efectele faptei drepte,\\
zicând: „Nu se va alege nimic din asta.”\\
La fel cum prin căderea treptată\\
a picurilor de ploaie\\
ulciorul de apă se umple,\\
astfel, în timp, cei înțelepți\\
devin preaplini cu bunătate.


\verseref{123}
La fel cum un negustor\\
cu încărcătură prețioasă\\
evită amenințările\\
și acei care iubesc viața evită otrava,\\
așa și tu să eviți faptele rele.


\verseref{124}
O mână fără o rană deschisă\\
poate purta otravă\\
și să rămână \\
liberă de vătămare;\\
la fel, răul nu are consecințe\\
pentru cei ce nu îl fac.


\verseref{125}
Dacă rănești intenționat\\
o persoană inocentă,\\
care e pură și nevinovată,\\
răul ți se va întoarce,\\
asemenea prafului fin\\
aruncat în vânt.


\verseref{126}
Unii se renasc în oameni;\\
răufăcătorii se renasc în iad.\\
Cei ce fac bine se renasc în exaltare,\\
iar cei puri merg în tărâmul fără urme.


\verseref{127}
Nu există loc pe pământ\\
unde cineva se poate ascunde\\
de consecințele faptelor rele –\\
nici într-o peșteră de munte,\\
nici în ocean, nici în cer.


\verseref{128}
Nu există loc pe pământ\\
unde moartea să nu ajungă –\\
nici într-o peșteră de munte,\\
nici în ocean, nici în cer.
