
\chapter{Fericire}


\verseref{197}
Cât timp te afli printre\\
cei ce urăsc\\
și sălășluiești fără de ură –\\
aceasta este cu adevărat fericirea.


\verseref{198}
Cât timp te afli printre\\
cei ce sunt tulburați\\
și sălășluiești fără de tulburare –\\
aceasta este cu adevărat fericirea.


\verseref{199}
Cât timp te afli printre\\
cei ce sunt lacomi\\
și sălășluiești fără de lăcomie –\\
aceasta este cu adevărat fericirea.


\verseref{200}
Sălășluim fericit,\\
fără angoasă;\\
ca ființele radioase\\
din tărâmurile celeste\\
sărbătorim încântarea.


\verseref{201}
Victoria duce la ură\\
căci înfrânții suferă.\\
Cei pașnici trăiesc fericiți,\\
dincolo de victorie și înfrângere.


\verseref{202}
Nu există niciun alt foc precum voluptatea,\\
nicio altă suferință precum ura,\\
nicio altă durere precum povara atașării,\\
nicio altă bucurie precum pacea adusă de liberare.


\verseref{203}
Pofta e cea mai mare nenorocire,\\
condiționarea e cea mai mare sursă de disperare.\\
Cei înțelepți, văzând lucrurile ca atare,\\
înțeleg că liberarea e cea mai mare bucurie.

\verseref{204}
O minte sănătoasă este cel mai mare câștig.\\
Împăcarea cu sine este cea mai mare avere.\\
Nădejdea este cel mai bun prieten.\\
Libertatea necondiționată este cea mai mare exaltare.


\verseref{205}
Gustând din savoarea solitudinii\\
și din nectarul păcii,\\
cei ce beau din fericirea care e esența realității\\
își duc veacul fără a se teme de rău.


\verseref{206}
It is always a pleasure\\
not to have to encounter fools.\\
It is always good to see noble beings,\\
and a delight to live with them.


\verseref{207}
Compania ignoranților e searbădă\\
și dureroasă de fiecare dată,\\
ca și cum te-ar înconjura inamicii;\\
însă asocierea cu cei înțelepți\\
înseamnă a te simți ca acasă.


\verseref{208}
Urmărește căile pe care le iau\\
cei ce sunt statornici,\\
deslușitori, puri și conștienți,\\
asemenea lunii care urmează\\
calea stelelor.
