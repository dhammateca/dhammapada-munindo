
\chapter{A Note on the Text}


\emph{A Dhammapada for Contemplation} is a contemporary rendering of an ancient text; but what of the original Dhammapada? The Buddha lived and taught in India two and a half millennia ago. He wrote nothing, and his teachings were memorised and passed on orally by his earliest followers. An enormous body of material was preserved in this way, although different versions arose in the various Buddhist schools, which flourished in India after the Buddha’s \emph{parinibbana}. The collection of verses known as the Dhammapada was composed, probably sometime in the 3rd century BCE, in a language now known as Pali, by members of the ‘Theravada’, or ‘School of the Elders’. We know of three other versions, written down later in other Indian languages, but they survive only partially, or as fragments rescued from archaeological finds, or as translations in Tibetan and Chinese. Those other versions mostly contain the same material, but there are many variations.

What we may think of as ‘the’ Dhammapada, then, is one version—the shortest and probably the earliest one—of a work with a lively history. It was taken to Sri Lanka later in the 3rd century BCE by the Theravadins, in its original Indian language, and written down along with the rest of the Pali canon in the 1st century BCE; it comes down to us today as preserved and commented upon by the Theravada tradition. It was the first Buddhist text to be printed in Europe, in 1855, along with a Latin translation, and has been translated into English at least 30 times. It is without doubt the best-known early Buddhist scripture, and to this day Sri Lankan monks will memorise it before their ordinations.

The reason for this popularity is not hard to discover. The Dhammapada is essentially a compilation of sayings, pithy, poetic and direct, that speak to all, not just to monk or nun or scholar. About half of them are found in other parts of the Pali canon, and although they are roughly grouped into chapters sharing a particular theme, most seem to have been chosen for their individual qualities, as sayings and sequences of sayings rich in wisdom to be savoured. Many of the sayings are concerned with ethics, with what is good and foolish and in what each results. Some of these are not, in fact, especially Buddhist, but are found also in other Indian religious texts, such as the great epic, the \emph{Mahabharata}. That is, they belong to a wider tradition of wisdom, of which we have our own versions in the west.

Beyond this basic human wisdom, however, the sayings of the Dhammapada concern the Way to “freedom from all limitation: liberation, true security” (verse 23), which is the particular teaching of the Buddha. They do not present this teaching in the form of doctrine, but instead offer a myriad hints, approaches and directions of investigation, from descriptions of the “great beings” (chapter 23) who have reached the other shore of existence, to compelling comparisons of the paths of “appreciative awareness” and “heedlessness” (verse 21). The appeal of the Dhammapada is greatly enhanced, however, by the similes, metaphors and poetic images lavishly employed to illustrate the meaning of the sayings; for instance, the influence of the wise is described as a light “like that of the moon emerging from clouds” (verse 172). For good measure, the Dhammapada also includes the Buddha’s ‘Song of Victory’, the poem he uttered, it is said, just after he had gained enlightenment (verses 153-4).

The sayings and poems of the Dhammapada were written in verse consisting of four- and six-line stanzas, with lines of eight or eleven syllables. Such forms belong to ancient Indian literary tradition, and lend to the text a distinctive rhythm. This rhythmic constraint necessitated the finding of words of the right number of syllables to fit the lines; the effect of which is a multiplicity and vigour in the language not found in the prose scriptures. There is therefore a formality in the Pali verse, which, like most poetry, is untranslatable. If we add to this untranslatability the demands of reproducing in English the epigrammatic and suggestive quality of the sayings, it is clear that a formally definitive translation of the Dhammapada is going to be difficult. For this reason, it is necessary to read the text in different translations, comparing the different emphases and resonances found in each, to come to a clear appreciation of the truth offered in each verse.

We could view the sayings of the Dhammapada as photographic negatives, as very old artefacts that miraculously contain within them the profound utterances of the Buddha. Such negatives can be used to make prints, in a modern language like English, which reproduce those ancient sayings. The attempts of scholarly translators to provide strictly objective, formally accurate prints from those old negatives is worthy and necessary, but the objective method does not necessarily transmit all that it was hoped would be captured by the original photographs, those few words of verse. Ajahn Munindo, in his rendering of the text, offers a more personal printing from each negative; one that is designed to transmit, not the objective meaning of each saying, but something relevant and useful for a person in search of wisdom today.

His method was this: using several respected translations, he ascertained what to his mind was the spirit or essence of each saying. This he did with the aid of the traditional story associated with each verse, which gives an account of the occasion and significance of its utterance. These stories were without doubt composed later than the text itself, and have in general a legendary character like that of the \emph{Jataka} tales, but nevertheless create an ambience, a mythic context, which gives access to an important dimension of what each saying means. Having thus ascertained the spirit of each verse, he sought an expression in contemporary language, which might transmit effectively its spiritual impulse.

Being an ancient Indian text, the Dhammapada abounds in references to monsoons, elephants, jungles, villages, brahmins and so on. While this is wonderfully evocative of life in the Buddha’s time, it distracts our attention from the point of the sayings; and therefore Ajahn Munindo’s version is, on the whole, rendered into a contemporary idiom relevant to the modern worldview. The \emph{birana} grass that grows fast after monsoon rains, for instance, is reinterpreted as weeds fed with fertiliser (verse 335), and the Indian ascetic’s matted hair, his antelope skin and his lifestyle are rendered as “outer adornments and pseudo-spiritual preoccupations” (verse 394). Most references to rebirth, and to the hell and heaven realms of Buddhist cosmology, have been replaced by psychological renderings of more relevance today. The references to the wandering mendicant lifestyle of the Buddhist monastic sangha, dependent as it was and is on a village-based agrarian culture, have been modified.

There are other special features of Ajahn Munindo’s version. He has not reproduced the gendered language of the original, so that his rendering hopes to address all readers. He has ignored much of the word play that is crucial to the original poetry, but at the benefit of much greater fluidity and naturalness. Although no attempt has been made to reproduce the formal qualities of the original, some of the beauty of the old text’s poetry emerges through the freedom enabled by an interpretive rendering (see, for instance, verse 377).

\bigskip

{\raggedleft
Thomas Jones Ph.D.\\
Cambridge
\par}
