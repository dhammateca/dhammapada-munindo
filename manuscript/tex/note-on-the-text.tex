
\chapter{O notă asupra textului}


\emph{O Dhammapada pentru contemplație} reprezintă o interpretare contemporană a unui text antic; însă cum rămâne cu Dhammapada originală? Buddha a trăit și a propovăduit în India, în urmă cu două milenii și jumătate. Nu a scris nimic, iar învățăturile sale au fost memorate și transmise prin viu grai de către primii săi discipoli. Prin această modalitate s-a păstrat un volum uriaș de material, deși au rezultat versiuni diferite în multiplele școli budiste care au apărut în număr mare în India după \emph{parinibbana} lui Buddha. Colecția de versuri cunoscute sub denumirea de Dhammapada a fost compusă, probabil, cândva în secolul al III-lea î.e.n., într-o limbă cunoscută în prezent drept pali, de către membri ai „Theravada” sau „Doctrina Vârstnicilor”. Știm de alte trei versiuni scrise ulterior în alte limbi din India, însă acestea au supraviețuit doar parțial sau sub formă de fragmente salvate din descoperiri arheologice ori ca traduceri în tibetană și chineză. Celelalte versiuni includ, în mare parte, același material, însă variațiile sunt numeroase.

Drept urmare, concepția noastră asupra Dhammapada „veritabilă” înseamnă o versiune – cea mai scurtă și, probabil, cea mai timpurie – a unei lucrări cu o istorie vie.
Textul a fost dus în Sri Lanka în a doua parte a secolului al III-lea î.e.n. de către adepții Theravada în originalul din limba indianăși apoi pus pe hârtie laolaltă cu canonul pali în secolul I î.e.n.; ne revine în prezent sub forma păstrată și comentată conform tradiției Theravada.
A fost primul text budist tipărit în Europa, în 1855, odată cu o traducere în latină. Textul a fost tradus în engleză de cel puțin 30 de ori. Fără îndoială, reprezintă cea mai cunoscută scriptură budistă timpurie și, chiar și în zilele noastre, călugării din Sri Lanka o memorează înainte de hirotonisire.

Motivul ce stă la baza acestei popularități nu este greu de identificat. Dhammapada reprezintă, în fond, o colecție de maxime pline de conținut, poetice și directe, care se adresează tuturor, nu doar călugărilor, călugărițelor sau cărturarilor. În jur de jumătate dintre acestea se regăsesc în alte părți ale canonului pali și, chiar dacă sunt grupate pe capitole pe o temă anume, majoritatea par să fi fost alese în funcție de calitățile lor individuale, drept maxime și secvențe de maxime pline de înțelepciune de apreciat. Multe dinte aceste maxime vizează etica, și anume ce este bine, ce este nesăbuit și care sunt consecințele aferente. Unele dintre acestea nu sunt, de fapt, îndeosebi budiste, ci se regăsesc și în alte texte religioase din India, cum ar fi epopeea impresionantă \emph{Mahabharata}. Mai exact, aparțin de o tradiție mai amplă a înțelepciunii, având și noi versiuni proprii în vest.

Cu toate acestea, dincolo de această înțelepciune umană de bază, maximele din Dhammapada fac referire la Calea către „libertatea din toate limitările: liberare, siguranță adevărată” (versetul 23), aceasta fiind anume învățătura lui Buddha. Această învățătură nu este prezentată sub formă de doctrină ci, în schimb, sunt oferite numeroase sugestii, abordări și indicații de studiu, de la descrierile „ființelor mărețe” (capitolul 23) care au ajuns pe celălalt țărm al existenței, până la comparații convingătoare dintre „băgarea de seamă” și „nebăgarea de seamă” (versetul 21). Farmecul Dhammapadei este sporit, totuși, de comparații, metafore și imagini poetice la care se face apel cu generozitate pentru a ilustra sensul maximelor; de exemplu, influența celor înțelepți este descrisă ca o lumină: „precum luna ivindu-se dintre nori” (versetul 172). Mai mult, Dhammapada include și „Cântecul victoriei” al lui Buddha, poemul rostit, se spune, imediat după ce a atins iluminarea (versetele 153-4).

Maximele și poemele Dhammapadei au fost scrise în versete formate din strofe de patru și șase rânduri, rânduri cu câte opt sau șapte silabe. Aceste forme sunt reprezentative tradiției literare antice din India, oferind textului un ritm aparte. Această constrângere ritmică a impus identificarea cuvintelor cu numărul corect de silabe pentru a încăpea pe rânduri; efectul acestui demers este o varietate și o vigoare care nu se regăsesc în scripturile sub formă de proză. Există, așadar, un aspect formal în ceea ce privește versetele pali. Acestea sunt, la fel ca majoritatea poeziilor, imposibil de tradus. Iar dacă adăugăm acestei imposibilități de traducere și cerințele reproducerii în engleză a calităților sugestive, de epigramă a maximelor, devine limpede că o traducere finală din punct de vedere formal a Dhammapadei va fi dificilă. Din acest motiv, este necesară citirea mai multor traduceri a textului, comparând diferitele emfaze și rezonanțe din fiecare pentru a putea aprecia în mod distinct adevărul transmis de fiecare verset.

Am putea privi maximele Dhammapadei ca pe niște negative foto, niște artefacte foarte vechi care conțin în mod miraculos spusele profunde ale lui Buddha. Aceste negative pot fi folosite pentru a tipări exemplare, într-o limbă modernă cum este engleza, ce reproduc acele maxime antice. Încercările traducătorilor învățați de a furniza exemplare strict obiective și exacte din punct de vedere formal pe baza acestor negative vechi sunt notabile și necesare, însă metoda obiectivă nu transmite neapărat toate aspectele despre care se spera că vor fi capturate prin fotografiile originale – acele câteva cuvinte sub formă de verset. Prin interpretarea sa, Ajahn Munindo oferă un exemplar mai personal pe baza fiecărui negativ, un exemplar conceput pentru a transmite nu sensul obiectiv al fiecărei maxime, ci un element relevant și util pentru o persoană aflată în prezent în căutarea înțelepciunii.

Metoda sa a fost următoarea: folosind mai multe traduceri respectate, a stabilit ceea ce minții sale i s-a părut ca fiind spiritul sau esența fiecărei maxime. A procedat astfel cu ajutorul poveștii tradiționale asociate fiecărui vers, care descrie ocazia și semnificația enunțării sale. Fără îndoială, aceste povești au fost compuse după textul în sine și au, în general, un caracter legendar precum poveștile \emph{Jataka}, însă, cu toate acestea, creează o atmosferă, un context mitic, care oferă acces la o dimensiune importantă a sensului fiecărei maxime. Stabilind astfel spiritului fiecărui verset, a căutat o expresie în limbajul contemporan care ar putea transmite efectiv impulsul său spiritual.

Reprezentând un text indian antic, în Dhammapada se găsesc din belșug referințe la musoni, elefanți, jungle, sate, brahmani ș.a.m.d. Cu toate că evocă în mod nemaipomenit viața din perioada lui Buddha, ne distrag atenția de la semnificația maximelor; drept urmare, versiunea lui Ajahn Munindo este, per ansamblu, redată într-un limbaj contemporan și relevant pentru perspectiva modernă asupra lumii. Iarba birana care crește rapid după ploile din anotimpul ploios, de exemplu, este reinterpretată drept buruieni fertilizate (versetul 335), iar părul încâlcit și ascetic al indianului, pielea sa de antilopă și stilul său de viață sunt redate drept „podoabele exterioare și preocupările pseudospirituale” (versetul 394). Majoritatea referirilor la renaștere, la iad și la tărâmurile cerești din cosmologia budistă au fost înlocuite cu reinterpretări psihologice mai relevante în prezent. Referirile la stilul de viață în pribegie, cerând de pomană al membrilor sangha monastic budist, ce a depins și depinde în continuare de o cultură rurală și agrară, au fost modificate.

Versiunea lui Ajahn Munindo are alte trăsături speciale. Nu a reprodus limbajul cu gen al textului original, astfel încât interpretarea sa speră să vizeze toți cititorii. A ignorat o mare parte din jocurile de cuvinte cruciale pentru poezia originală, însă cu avantajele fluidității și naturaleței mult sporite. Deși nu s-a încercat reproducerea calităților formale ale originalului, o parte din frumusețea poeziei textului vechi reiese în urma libertății permise de o tălmăcire interpretativă (vezi, de exemplu, versetul 377).

\bigskip

{\raggedleft
Thomas Jones Ph.D.\\
Cambridge
\par}
