
\chapter{The Just}


\verseref{256}
Making an arbitrary decision\\
does not amount to justice.\\
The wise decide the case\\
after considering arguments\\
for and against.


\verseref{257}
By making decisions\\
based on truth and fairness\\
one safeguards the law\\
and is called righteous.


\verseref{258}
Those who speak much\\
are not necessarily possessed of wisdom.\\
The wise can be seen\\
to be at peace with life\\
and free from all enmity and fear.


\verseref{259}
Though one's knowledge\\
may be limited,\\
if understanding and conduct\\
rightly accord with the Way,\\
one is to be considered\\
well-versed in Dhamma.


\verseref{260}
Having grey hair\\
does not make you an elder;\\
ripe in years maybe,\\
but perhaps pointlessly so.


\verseref{261}
One who is truthful,\\
virtuous, impeccable in conduct,\\
free from all stains and wise\\
can be called an elder.


\verseref{262 -263}
Those who are envious,\\
stingy and manipulative\\
remain unappealing despite\\
good looks and eloquent speech.\\
But those who have freed themselves\\
from their faults\\
and arrived at wisdom\\
are attractive indeed.


\verseref{264}
Shaving your head\\
does not make you a renunciate\\
if you are still full of\\
recklessness and deceit.\\
How could someone possessed\\
by craving and lust\\
be considered a renunciate?


\verseref{265}
You become a monk or nun\\
by letting go of all evil,\\
by renouncing all\\
unwholesomeness,\\
both great and small.


\verseref{266}
You are not a monk or nun\\
because you depend\\
on others for food,\\
but by submitting yourself\\
wholeheartedly\\
to the training of body,\\
speech and mind.


\verseref{267}
You become a monk or nun\\
by seeing through this world\\
with understanding,\\
by rising above good and bad\\
and living a life of purity\\
and contemplation.


\verseref{268-269}
Silence does not denote profundity\\
if you are ignorant and untrained.\\
Like one holding scales\\
a sage weighs things up,\\
wholesome and unwholesome,\\
and comes to know\\
both the inner and outer worlds.\\
Therefore the sage is called wise.


\verseref{270}
Those who still cause harm\\
to living beings\\
cannot be considered as attained.\\
Those who are attained\\
maintain a harmless demeanour\\
towards all beings.


\verseref{271-272}
Do not rest contented\\
because you keep all the rules and regulations,\\
nor because you achieve great learning.\\
Do not feel satisfied because you\\
attain meditative absorption,\\
nor because you can dwell in\\
the bliss of solitude.\\
Only when you arrive\\
at the complete eradication\\
of all ignorance and conceit\\
should you be content.

