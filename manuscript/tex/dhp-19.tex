
\chapter{Cei drepți}


\verseref{256}
Luatul unei decizii arbitrare\\
nu duce la dreptate.\\
Cei înțelepți hotărăsc asupra cazului\\
după ce au în vedere argumentele pentru și împotriva.


\verseref{257}
Luând decizii în baza adevărului și echității,\\
insul protejează legea și se poate numi virtuos.


\verseref{258}
Cei ce vorbesc mult\\
nu sunt neapărat înzestrați cu înțelepciune.\\
Se poate vedea că înțelepții\\
sunt împăcați cu viața și fără dușmănie și teamă.


\verseref{259}
Deși cunoștințele unui ins ar putea fi limitate,\\
dacă înțelegerea și conduita sunt în conformitate cu Calea,\\
se consideră că insul este un bun cunoscător al Dhammei.


\verseref{260}
A avea părul cărunt\\
nu vă aduce în rândul venerabililor;\\
sunteți poate copți în privința anilor,\\
însă posibil fără sens.


\verseref{261}
Insul sincer, virtuos, de o conduită impecabilă,\\
eliberat de toate întinările și înțelept\\
poate fi numit venerabil.


\verseref{262 -263}
Cei invidioși, zgârciți și manipulatori\\
rămân neatrăgători în ciuda înfățișării\\
plăcute și vorbitului elocvent.\\
Însă cei care s-au eliberat de defecte\\
și au ajuns la înțelepciune\\
sunt cu adevărat atrăgători.


\verseref{264}
Rasul părului de pe cap\\
nu vă transformă în cei ce renunță\\
dacă rămâneți plini de nechibzuință și înșelăciune.\\
Cum poate cineva posedat de râvnă și voluptate\\
să fie în rând cu cei care renunță?


\verseref{265}
Deveniți călugări sau călugărițe\\
abandonând pe deplin răul,\\
renunțând la tot ce nu e benefic,\\
atât mare, cât și mic.


\verseref{266}
Nu sunteți călugări sau călugărițe\\
pentru că depindeți de alții pentru mâncare,\\
ci dedicându-vă cu toată inima\\
instruirii trupului, vorbitului și minții.


\verseref{267}
Deveniți călugări sau călugărițe\\
privind prin această lume cu înțelegere,\\
fiind dincolo de bine și de rău\\
și trăind o viață pură și contemplativă.


\verseref{268-269}
Liniștea nu denotă profunzime\\
dacă sunteți ignoranți și neinstruiți.\\
Ca și cum ar ține în mână o balanță\\
un învățat cântărește lucrurile,\\
prielnice și neprielnice,\\
și ajunge să cunoască atât lumile interioare, cât și exterioare.\\
Astfel învățatul se numește înțelept.


\verseref{270}
Cei ce încă fac rău ființelor vii\\
nu pot fi considerați nobili.\\
Cei nobili păstrează o comportare\\
inofensivă față de toate ființele.


\verseref{271-272}
Nu fiți mulțumiți\\
doar pentru că respectați toate regulile și normele,\\
nici pentru că dobândiți învățăminte vaste.\\
Nu vă simțiți satisfăcuți\\
doar pentru că ați dobândit\\
absorbția în meditație,\\
nici pentru că vă puteți cufunda\\
în exaltarea solitudinii.\\
Doar când ajungeți la eradicarea completă\\
a oricărei urme de ignoranță și trufie\\
trebuie să fiți satisfăcuți.
