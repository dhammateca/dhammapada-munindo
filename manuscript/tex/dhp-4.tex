
\chapter{Flori}


\verseref{44}
Cine sunt\\
cei care pot să vadă sincer\\
așa cum sunt\\
acest pământ, acest corp,\\
tărâmurile iadului și tărâmurile paradisului?\\
Cine poate discerne\\
bine predicata Cale a Înțelepciunii,\\
la fel cum ochiul unui florar iscusit\\
poate alege flori perfecte?

\verseref{45}
Sunt ei\\
cei care umblă conștient pe Cale\\
care văd cum sunt cu adevărat\\
acest pământ, acest corp,\\
tărâmurile iadului și tărâmurile paradisului.\\
Ei sunt aceia care pot discerne\\
bine predicata Cale a Înțelepciunii.


\verseref{46}
Cunoaște corpul\\
ca fiind trecător ca spuma,\\
un miraj.\\
Floarea pasiunii senzuale\\
are un spin ascuns.\\
Vezi asta și treci dincolo de moarte.


\verseref{47}
Așa cum o inundație din senin\\
poate mătura din cale un sat care doarme,\\
așa moartea îi poate distruge\\
pe cei care caută doar florile\\
plăcerilor senzuale banale.


\verseref{48}
A viețui cu frustrare și nemulțumire,\\
dar totuși a căuta doar florile\\
plăcerilor senzuale banale\\
îl duce pe ins sub stăpânirea distrugătorului.


\verseref{49}
Precum o albină care,\\
adunând nectar,\\
nu rănește sau perturbă\\
culoarea și parfumul florii\\
astfel și cei înțelepți\\
se mișcă prin lume.


\verseref{50}
Nu poposi asupra defectelor\\
și neajunsurilor altora;\\
în schimb, caută claritate\\
în privința alor tale.


\verseref{51}
Precum o floare frumoasă\\
lipsită de mireasmă dezamăgește,\\
așa sunt și cuvintele înțelepte\\
fără fapte drepte.



\verseref{52}
Precum o floare frumoasă\\
cu o mireasmă îmbietoare încântă,\\
așa e și vorbitul înțelept și dulce\\
când e însoțit de fapte drepte.


\verseref{53}
Așa cum multe ghirlande pot fi făcute\\
dintr-o adunătură de flori,\\
la fel multe iscusituri pot fi făcute\\
pe durata acestei vieți omenești.


\verseref{54}
Mireasma florilor sau a lemnului de santal\\
bate doar în direcția vântului predominant,\\
dar mireasma virtuții\\
străbate toate direcțiile.


\verseref{55}
Mireasma virtuții\\
de departe-ntrece\\
mireasma florilor\\
sau a lemnului de santal.


\verseref{56}
Aroma lemnului de santal\\
și mirosul florilor\\
aduc doar puțină bucurie\\
în comparație cu mireasma virtuții,\\
care inundă chiar și tărâmurile cerești.


\verseref{57}
Nu e posibil ca Mara să-i găsească\\
pe cei care-și duc veacul în luciditate,\\
care prin cunoaștere perfectă sunt eliberați\\
și trăiesc în virtute.

\verseref{58-59}
La fel cum un lotus frumos și dulce mirositor\\
poate crește dintr-o grămadă\\
de gunoi aruncat,\\
strălucirea unui adevărat\\
discipol al lui Buddha\\
întrece umbrele întunecate\\
aruncate de ignoranță.
