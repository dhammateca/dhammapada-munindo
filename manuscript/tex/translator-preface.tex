
\chapter{Prefața traducătorului}

Am dat peste această Dhammapada, în interpretarea Venerabilului Ajahn Munindo, pe durata unei șederi la Mănăstirea Muttodaya. Am început să traduc câteva dintre versuri și, la întoarcere, le-am trimis prietenilor. Văzând interesul pe care l-au stârnit, am decis să o traduc. Totuși, e important de ținut în minte faptul că autorul consideră „O Dhammapada pentru Contemplație” mai degrabă o interpretare decât o traducere.

Am consultat și alte traduceri înainte de a începe, dar această interpretare s-a dovedit cea mai bogată în sensuri inteligibile pentru mine, având un spirit contemporan și dovedindu-se a fi și un manual de exerciții pentru cultivarea minții.

Așadar, pentru traducerea multor cuvinte și expresii, am preferat semnificația tehnică din punctul de vedere al temelor de meditație budiste, în defavoarea sensurilor poetice; introspecție (\emph{insight}), echidistanță (\emph{equanimity}), băgare de seamă (\emph{heedfulness}), luciditate cufundată în corp (\emph{meditation practice focused in the body}) etc.

În locurile unde limba română s-a dovedit mai expresivă, am încercat fructificarea acestei situații. În alte locuri, în schimb, originalul s-a dovedit mai expresiv, iar traducerea a avut nevoie de un artificiu pentru a păstra înrudirea fonetică: \emph{not inflated} ~ neumflat în pene (versetul 363).

Intervenția cea mai notabilă este în versetul 384, unde am subliniat că este vorba despre două dhamma pentru a evita ca versetul să rămână criptic pentru cineva nefamiliarizat cu învățăturile budiste.

M-am bucurat de ajutorul unei traducătoare profesioniste care, pe lângă corectură, s-a îngrijit să ferească textul de ridicol și greșeli gramaticale. Am abordat împreună procesul de traducere respectând în cel mai înalt grad posibil varianta în limba engleză, atât în ceea ce privește forma, cât și fondul. Versiunea fiind deja adaptată în mare măsură contemporanului, nu am introdus modificări semantice. În cazul referirilor la persoane am folosit cu prioritate pluralul și o variantă neutră, fără gen, pentru a ne adresa la modul general, iar tonul textului să nu vizeze o persoană anume.

Am consultat traduceri alternative în limba engleză din limba pāli („The Dhammapada: A Translation”, Thanissaro Bhikkhu, Access to Insight, 1997, „The Dhammapada: The Buddha's Path of Wisdom”, Buddhist Publication Society, Acharya Buddharakkhita, 1985, „The Dhammapada: Verses and Stories Translated”, Daw Mya Tin, M.A, Burma Pitaka Association, 1986). Am consultat și demersurile colegilor traducători în franceză („Le Dhammapada”, Ed. Les Deux Océans, 1989), spaniolă din Argentina („El Dhammapada”, Ed. Hastinapura, 2004), traduse la rândul lor din diferite versiuni în engleză (de exemplu, varianta din franceză se bazează pe traducerea din pāli în engleză a Venerabilului Maha Thera Narada).

Am respectat, în mare parte, organizarea pe rânduri a versetelor, existând bineînțeles abateri pentru a reda topica din limba română. De asemenea, pe alocuri, am izolat anumite sintagme pe rânduri pentru a evidenția accentul pus asupra conceptului său sau pentru a reda textul cu o
muzicalitate specifică limbii noastre (versetul 4, de exemplu).

Aceasta esta prima ediția a traducerii, care se bazează pe ediția a cincea a originalului.

Ne cerem iertare cititorului pentru sensurile de care l-am văduvit, iar pentru cele care sunt clare, îl încurajăm să le traducă mai departe, în inima sa.

\bigskip

{\raggedleft
Mircea Mare, Diana D.\\
www.dhammadha.ro
\par}
