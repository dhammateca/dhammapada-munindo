
\chapter{Elefantul}


\verseref{320}
Așa cum un elefant în luptă\\
ține piept săgeților,\\
aleg să îndur\\
atacurile verbale din partea celorlalți.


\verseref{321}
Caii dresați cum se cuvine\\
pot fi de încredere în mulțimi\\
și sunt destinați a fi călăriți de regi.\\
Indivizii care s-au instruit să îndure abuzul\\
vor fi de prețuit pretutindeni.

\verseref{322}
Caii sau elefanții dresați cum se cuvine\\
sunt impresionați;\\
însă mult mai impresionați\\
sunt indivizii care s-au îmblânzit.


\verseref{323}
Nu pe un animal bine dresat\\
veți putea călări către tărâmul liberării;\\
numai purtați de un autocontrol bine instruit\\
veți ajunge acolo.


\verseref{324}
Când este capturat, legat și în călduri,\\
elefantul sălbatic este agitat,\\
de necontrolat și nu își mănâncă mâncarea.\\
Tânjește după pădurea nativă de unde se trage.

\verseref{325}
Necopt e acela care mănâncă în exces\\
și se dedă lenei, iar apoi,\\
simțindu-se vlăguit,\\
se lăfăie dormind asemenea unui\\
porc supradimensionat;\\
acest lucru prevestește suferința perpetuă.


\verseref{326}
Mintea mea, cândva sălbatică\\
și de nestăpânit –\\
acum o controlez\\
așa cum un conducător de elefanți\\
controlează cu cârligul său\\
un elefant în călduri.


\verseref{327}
Așa cum un elefant\\
se urnește cu hotărâre dintr-o mlaștină,\\
înălțați-vă având drept inspirație\\
atenția cultivată.

\verseref{328}
Dacă găsești un companion de nădejde,\\
integru și înțelept,\\
vei depăși toate pericolele\\
în companie veselă și grijulie.


\verseref{329}
Însă dacă nu poți găsi\\
un companion de nădejde,\\
integru și înțelept,\\
atunci, asemenea unui rege\\
ce lasă în urmă un tărâm părăsit\\
sau unui elefant ce rătăcește de unul\\
singur în pădure,\\
înaintează singur.


\verseref{330}
O viață singuratică inofensivă\\
trăită în tihnă,\\
asemenea celei a elefantului retras în pădure,\\
este mai bună\\
decât compania zădarnică a necopților.


\verseref{331}
Compania potrivită a prietenilor înseamnă\\
bunătate.\\
Puținătatea nevoilor înseamnă bunătate.\\
Strângerea de virtute\\
până la sfârșitul vieții înseamnă bunătate.\\
A se lipsi\\
de toată suferința înseamnă bunătate.


\verseref{332}
A-și deservi cu cinste părinții înseamnă\\
bunătate.\\
Ajutorarea celor ce renunță înseamnă bunătate.\\
Cinstirea ființelor trezite înseamnă bunătate.


\verseref{333}
Păstrarea virtuții până la bătrânețe\\
înseamnă bunătate.\\
Păstrarea unei credințe autentice înseamnă\\
bunătate.\\
Apariția introspecției înseamnă bunătate.\\
Renunțarea la rău înseamnă bunătate.
