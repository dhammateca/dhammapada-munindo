
\chapter{Iad}


\verseref{306}
Minciuna duce la supărare.\\
Tăinuirea acțiunilor greșite\\
duce la supărare.\\
Aceste două fapte de înșelăciune\\
duc ființele\\
la aceeași stare de suferință.


\verseref{307}
Cei ce poartă robele\\
unuia care a renunțat,\\
dar nutresc rău și sunt nestăpâniți\\
trec la o stare de supărare.


\verseref{308}
Celui ce a renunțat\\
i-ar fi mai bine înghițind fier topit\\
decât să trăiască din ofrande\\
dobândite necinstit.


\verseref{309}
Tristețea adunată,\\
somnul neliniștit,\\
vinovăția și remușcarea\\
sunt povara celui adulter.


\verseref{310}
Scurtă este încântarea\\
cuplului adulter temător\\
căci nu pot urma decât consecințe dureroase.


\verseref{311}
Așa cum iarba \emph{kusa} prinsă nepotrivit\\
rănește mâna celui ce o apucă,\\
la fel traiul unuia care a renunțat\\
îi rănește pe cei ce se instruiesc nepotrivit.


\verseref{312}
Faptele comise nechibzuit,\\
practicile făcute greșit,\\
viața sfântă trăită cu depravare\\
aduc puține beneficii sau niciunul.


\verseref{313}
Dacă aveți ceva de făcut,\\
faceți-o cum se cuvine,\\
cu energie și dedicare;\\
viața unuia ce a renunțat\\
trăită fără băgare de seamă\\
nu face decât să ridice praful.


\verseref{314}
Faptele dureroase\\
mai bine rămân nefăcute\\
deoarece conduc mereu la remușcări.\\
Faptele inofensive\\
e mai bine să fie făcute\\
deoarece nu vor fi urmate de regrete.


\verseref{315}
Așa cum un oraș de graniță trebuie apărat\\
cu grijă,\\
păziți-vă atât pe dinăuntru, cât și pe dinafară;\\
clădiți-vă o apărare cu înțelepciune și în timp.\\
Dacă nu vă îngrijiți la momentul cuvenit de aceste lucruri,\\
va urma o tristețe mare.


\verseref{316}
Percepțiile denaturate\\
care nasc stări de rușine\\
față de ceea ce nu e rușinos\\
sau indiferență\\
față de ceea ce e rușinos\\
pot duce la coborârea ființelor în iad.


\verseref{317}
Percepțiile denaturate\\
care nasc stări de teamă\\
față de ceea ce nu e amenințător\\
sau indiferență în fața\\
a ceea ce e amenințător\\
pot duce la coborârea ființelor în iad.


\verseref{318}
Percepțiile denaturate\\
care duc la vederea binelui drept rău\\
și a răului drept bine\\
duc la dezintegrarea ființelor.


\verseref{319}
Cel ce vede limpede\\
și distinge ceea ce e greșit drept greșit\\
și ceea ce e pur drept pur\\
poate duce ființele dincolo de obidă.
