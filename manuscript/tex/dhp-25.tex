
\chapter{The Renunciate}


\verseref{360}
It is good to restrain the eye.\\
It is good to restrain the ear.\\
It is good to restrain the nose.\\
It is good to restrain the tongue.


\verseref{361}
It is good to be restrained in body.\\
It is good to be restrained in speech.\\
It is good to be restrained in mind.\\
It is good to be restrained in everything.\\
The renunciate who is restrained\\
in every way will realize freedom from suffering.


\verseref{362}
One who is rightly disciplined\\
in all actions,\\
who is composed,\\
contented and delights in\\
solitary contemplation --\\
such is a renunciate.


\verseref{363}
It is pleasing to hear\\
the words of a renunciate\\
who is wise, not inflated,\\
whose mind is composed\\
and whose speech is contained\\
and clear in meaning.


\verseref{364}
One who abides in Dhamma,\\
who delights in Dhamma,\\
who contemplates Dhamma,\\
who memorizes Dhamma\\
does not lose the Way.


\verseref{365}
Bemoaning your own lot\\
or envying the gains of others\\
obstructs peace of mind.


\verseref{366}
But being contented\\
even with modest gains,\\
pure in livelihood and energetic,\\
you will be held in high esteem.


\verseref{367}
True renunciates\\
regard the entire body-mind\\
without any thought of ’I’ or ’mine’\\
and are devoid of longing\\
for what they do not have.


\verseref{368}
A renunciate who\\
abides in loving-kindness,\\
with a heart full of devotion\\
for the Buddha's teaching,\\
will find peace, stillness and bliss.


\verseref{369}
Bale out the water from your boat;\\
cut loose from the defiling passions\\
of lust and hatred;\\
unencumbered, sail on\\
towards liberation.


\verseref{370}
One who has\\
cut off coarse attachments,\\
cut off subtle attachments,\\
who cultivates the spiritual faculties,\\
is the one who finds freedom\\
from delusion.


\verseref{371}
Be careful!\\
Do not neglect meditation,\\
nor allow the mind\\
to dwell on sensuality\\
lest you might heedlessly swallow\\
a red-hot ball of iron\\
and find yourself crying out,\\
“Why am I suffering?”


\verseref{372}
Concentration does not arise\\
without understanding,\\
nor understanding\\
without concentration.\\
One who knows both\\
approaches liberation.


\verseref{373}
A happiness transcending ordinary bliss\\
is experienced by those renunciates\\
who have entered into seclusion\\
with tranquil heart\\
and clear understanding of the Way.


\verseref{374}
When those who are wise\\
dwell in contemplation\\
on the transient nature\\
of the body-mind\\
and of all conditioned existence,\\
they experience joy and delight,\\
seeing through\\
to the inherently secure.


\verseref{375-376}
This then is the beginning\\
for a renunciate who takes up the training:\\
wisely control your faculties,\\
commit yourself to the instruction,\\
seek contentment;\\
cultivate the company of those\\
who support your aspiration\\
for energetic practice of the teachings.\\
The beauty of pure conduct\\
conditions whole-hearted well-being,\\
giving rise to complete\\
freedom from remorse.


\verseref{377}
As old flowers fall\\
from a jasmine plant,\\
let lust and hatred\\
fall away.


\verseref{378}
I call them the peaceful ones,\\
who are calm in body,\\
in speech and in mind,\\
and who are thoroughly purged\\
of all worldly obsessions.


\verseref{379}
Scrutinize yourself.\\
Examine yourself.\\
With right attention\\
to self assessment\\
you will live at ease.


\verseref{380}
We are our own protection;\\
we are indeed our own secure abiding;\\
how could it be otherwise?\\
So with due care\\
we attend to ourselves.


\verseref{381}
A monk, a nun, who cultivates\\
a joyous disposition\\
and is filled with\\
confidence in the Way\\
will find peace, stillness and bliss.


\verseref{382}
While still a youth, a renunciate\\
fully devoted to the Way\\
lights up the world\\
like the moon emerging from clouds.

