
\chapter{Cel ce renunță}


\verseref{360}
Este bine a-și abține ochii.\\
Este bine a-și abține urechile.\\
Este bine a-și abține nasul.\\
Este bine a-și abține limba.


\verseref{361}
Este bine a-și abține corpul.\\
Este bine a-și abține vorba.\\
Este bine a-și abține mintea.\\
Este bine a-și abține tot.\\
Cel ce renunță și se abține în toate privințele\\
va dobândi liberarea de suferință.


\verseref{362}
Insul disciplinat cum se cade\\
în toate acțiunile,\\
liniștit, mulțumit\\
și care se delectează\\
în contemplare singuratică –\\
acesta este cel ce renunță.


\verseref{363}
E plăcut a auzi cuvintele unuia ce renunță;\\
înțelept, neumflat în pene,\\
cu mintea liniștită\\
și cu vorba reținută\\
și cu sensul limpede.


\verseref{364}
Cel ce își duce veacul în Dhamma,\\
care se delectează în Dhamma,\\
care contemplă Dhamma,\\
care memorează Dhamma,\\
acela nu se va abate de la Cale.


\verseref{365}
A-ți deplânge soarta\\
sau a invidia avuțiile altora\\
stă în calea liniștii sufletești.


\verseref{366}
Însă fiind mulțumit\\
chiar și cu avuții modeste,\\
trăind cu puritate și energic\\
te vei bucura de un respect deosebit.


\verseref{367}
Cei ce renunță cu adevărat\\
se gândesc pe deplin la trup-minte\\
fără vreo referire la „eu” sau „al meu”\\
și nu își doresc ceea ce nu au.


\verseref{368}
Insul ce renunță\\
și își duce veacul cu bunătate pașnică\\
și o inimă plină de devotament\\
față de învățăturile lui Buddha\\
va găsi pace, liniște și exaltare.


\verseref{369}
Scoate-ți apa din barcă;\\
taie-ți legăturile de pasiunile pângăritoare\\
ale voluptății și urii;\\
neîmpovărat, navighează către liberare.


\verseref{370}
Insul care\\
a tăiat atașamentele brute,\\
a tăiat atașamentele subtile,\\
care-și cultivă facultățile spirituale\\
este cel care găsește eliberarea de decepție.


\verseref{371}
Aveți grijă!\\
Nu neglijați meditația\\
și nu permiteți minții să stăruiască\\
asupra senzualității\\
ca nu cumva să înghițiți fără băgare de seamă\\
o bilă roșie și încinsă de fier\\
și să vă întrebați: „De ce sufăr?”.


\verseref{372}
Concentrarea nu apare\\
fără înțelegere,\\
iar nici înțelegerea nu apare\\
fără concentrare.\\
Insul care le cunoaște pe ambele\\
se apropie de liberare.


\verseref{373}
O fericire dincolo de exaltarea obișnuită\\
este cea resimțită de cei ce renunță\\
care s-au retras în singurătate\\
cu inima liniștită\\
și cu o înțelegere limpede a Căii.


\verseref{374}
Atunci când cei înțelepți\\
trăiesc contemplând\\
asupra naturii efemere a\\
acestui corp-minte\\
și a oricărei existențe condiționate,\\
resimt bucurie și încântare\\
văzând calea către siguranța firească.


\verseref{375-376}
Acesta este atunci începutul\\
pentru cel ce renunță care începe instruirea:\\
controlează-ți facultățile,\\
dedică-te învățăturii,\\
caută împăcarea cu sine;\\
cultivă compania celor\\
care îți susțin aspirațiile\\
pentru practica energică a învățăturilor.\\
Frumusețea conduitei pure\\
condiționează starea de bine deplină,\\
ducând la eliberarea completă de remușcări.


\verseref{377}
Așa cum florile vechi\\
cad de pe o plantă de iasomie,\\
lasă voluptatea și ura\\
să cadă.


\verseref{378}
Îi numesc „cei pașnici”,\\
cu trupul calm, vorba calmă și mintea calmă\\
și care s-au descotorosit în profunzime\\
de toate obsesiile lumești.


\verseref{379}
Cercetează-te.\\
Examinează-te.\\
Acordând atenția corectă\\
evaluării de sine,\\
vei trăi în tihnă.


\verseref{380}
Noi suntem cei care ne protejăm;\\
noi suntem cu adevărat propriul nostru sălaș;\\
cum ar putea sta altfel lucrurile?\\
Așadar, cu atenția cuvenită, ne ocupăm de noi\\
așa cum un dresor s-ar îngriji de un pursânge.

\verseref{381}
Un călugăr, o călugăriță care cultivă\\
o dispoziție voioasă\\
și sunt plini de încredere în Cale\\
vor găsi pace, liniște și exaltare.


\verseref{382}
Cât timp este încă tânăr, cel ce renunță\\
și este complet devotat Căii\\
luminează lumea\\
asemenea lunii care iese din nori.
